% Options for packages loaded elsewhere
\PassOptionsToPackage{unicode}{hyperref}
\PassOptionsToPackage{hyphens}{url}
%
\documentclass[
  11pt,
  letterpaper,
  oneside,
  open=any]{scrbook}

\usepackage{amsmath,amssymb}
\usepackage{lmodern}
\usepackage{iftex}
\ifPDFTeX
  \usepackage[T1]{fontenc}
  \usepackage[utf8]{inputenc}
  \usepackage{textcomp} % provide euro and other symbols
\else % if luatex or xetex
  \usepackage{unicode-math}
  \defaultfontfeatures{Scale=MatchLowercase}
  \defaultfontfeatures[\rmfamily]{Ligatures=TeX,Scale=1}
  \setmainfont[]{Atkinson Hyperlegible}
  \setsansfont[]{Atkinson Hyperlegible}
\fi
% Use upquote if available, for straight quotes in verbatim environments
\IfFileExists{upquote.sty}{\usepackage{upquote}}{}
\IfFileExists{microtype.sty}{% use microtype if available
  \usepackage[]{microtype}
  \UseMicrotypeSet[protrusion]{basicmath} % disable protrusion for tt fonts
}{}
\makeatletter
\@ifundefined{KOMAClassName}{% if non-KOMA class
  \IfFileExists{parskip.sty}{%
    \usepackage{parskip}
  }{% else
    \setlength{\parindent}{0pt}
    \setlength{\parskip}{6pt plus 2pt minus 1pt}}
}{% if KOMA class
  \KOMAoptions{parskip=half}}
\makeatother
\usepackage{xcolor}
\usepackage[lmargin=1in,rmargin=1in,tmargin=1in,bmargin=1in]{geometry}
\setlength{\emergencystretch}{3em} % prevent overfull lines
\setcounter{secnumdepth}{5}
% Make \paragraph and \subparagraph free-standing
\ifx\paragraph\undefined\else
  \let\oldparagraph\paragraph
  \renewcommand{\paragraph}[1]{\oldparagraph{#1}\mbox{}}
\fi
\ifx\subparagraph\undefined\else
  \let\oldsubparagraph\subparagraph
  \renewcommand{\subparagraph}[1]{\oldsubparagraph{#1}\mbox{}}
\fi


\providecommand{\tightlist}{%
  \setlength{\itemsep}{0pt}\setlength{\parskip}{0pt}}\usepackage{longtable,booktabs,array}
\usepackage{calc} % for calculating minipage widths
% Correct order of tables after \paragraph or \subparagraph
\usepackage{etoolbox}
\makeatletter
\patchcmd\longtable{\par}{\if@noskipsec\mbox{}\fi\par}{}{}
\makeatother
% Allow footnotes in longtable head/foot
\IfFileExists{footnotehyper.sty}{\usepackage{footnotehyper}}{\usepackage{footnote}}
\makesavenoteenv{longtable}
\usepackage{graphicx}
\makeatletter
\def\maxwidth{\ifdim\Gin@nat@width>\linewidth\linewidth\else\Gin@nat@width\fi}
\def\maxheight{\ifdim\Gin@nat@height>\textheight\textheight\else\Gin@nat@height\fi}
\makeatother
% Scale images if necessary, so that they will not overflow the page
% margins by default, and it is still possible to overwrite the defaults
% using explicit options in \includegraphics[width, height, ...]{}
\setkeys{Gin}{width=\maxwidth,height=\maxheight,keepaspectratio}
% Set default figure placement to htbp
\makeatletter
\def\fps@figure{htbp}
\makeatother
\newlength{\cslhangindent}
\setlength{\cslhangindent}{1.5em}
\newlength{\csllabelwidth}
\setlength{\csllabelwidth}{3em}
\newlength{\cslentryspacingunit} % times entry-spacing
\setlength{\cslentryspacingunit}{\parskip}
\newenvironment{CSLReferences}[2] % #1 hanging-ident, #2 entry spacing
 {% don't indent paragraphs
  \setlength{\parindent}{0pt}
  % turn on hanging indent if param 1 is 1
  \ifodd #1
  \let\oldpar\par
  \def\par{\hangindent=\cslhangindent\oldpar}
  \fi
  % set entry spacing
  \setlength{\parskip}{#2\cslentryspacingunit}
 }%
 {}
\usepackage{calc}
\newcommand{\CSLBlock}[1]{#1\hfill\break}
\newcommand{\CSLLeftMargin}[1]{\parbox[t]{\csllabelwidth}{#1}}
\newcommand{\CSLRightInline}[1]{\parbox[t]{\linewidth - \csllabelwidth}{#1}\break}
\newcommand{\CSLIndent}[1]{\hspace{\cslhangindent}#1}

\usepackage{scrlayer-scrpage}
\rohead{Header text}
\rofoot{Footer text}
\makeatletter
\@ifpackageloaded{tcolorbox}{}{\usepackage[many]{tcolorbox}}
\@ifpackageloaded{fontawesome5}{}{\usepackage{fontawesome5}}
\definecolor{quarto-callout-color}{HTML}{909090}
\definecolor{quarto-callout-note-color}{HTML}{0758E5}
\definecolor{quarto-callout-important-color}{HTML}{CC1914}
\definecolor{quarto-callout-warning-color}{HTML}{EB9113}
\definecolor{quarto-callout-tip-color}{HTML}{00A047}
\definecolor{quarto-callout-caution-color}{HTML}{FC5300}
\definecolor{quarto-callout-color-frame}{HTML}{acacac}
\definecolor{quarto-callout-note-color-frame}{HTML}{4582ec}
\definecolor{quarto-callout-important-color-frame}{HTML}{d9534f}
\definecolor{quarto-callout-warning-color-frame}{HTML}{f0ad4e}
\definecolor{quarto-callout-tip-color-frame}{HTML}{02b875}
\definecolor{quarto-callout-caution-color-frame}{HTML}{fd7e14}
\makeatother
\makeatletter
\makeatother
\makeatletter
\@ifpackageloaded{bookmark}{}{\usepackage{bookmark}}
\makeatother
\makeatletter
\@ifpackageloaded{caption}{}{\usepackage{caption}}
\AtBeginDocument{%
\ifdefined\contentsname
  \renewcommand*\contentsname{Table of contents}
\else
  \newcommand\contentsname{Table of contents}
\fi
\ifdefined\listfigurename
  \renewcommand*\listfigurename{List of Figures}
\else
  \newcommand\listfigurename{List of Figures}
\fi
\ifdefined\listtablename
  \renewcommand*\listtablename{List of Tables}
\else
  \newcommand\listtablename{List of Tables}
\fi
\ifdefined\figurename
  \renewcommand*\figurename{Figure}
\else
  \newcommand\figurename{Figure}
\fi
\ifdefined\tablename
  \renewcommand*\tablename{Table}
\else
  \newcommand\tablename{Table}
\fi
}
\@ifpackageloaded{float}{}{\usepackage{float}}
\floatstyle{ruled}
\@ifundefined{c@chapter}{\newfloat{codelisting}{h}{lop}}{\newfloat{codelisting}{h}{lop}[chapter]}
\floatname{codelisting}{Listing}
\newcommand*\listoflistings{\listof{codelisting}{List of Listings}}
\makeatother
\makeatletter
\@ifpackageloaded{caption}{}{\usepackage{caption}}
\@ifpackageloaded{subcaption}{}{\usepackage{subcaption}}
\makeatother
\makeatletter
\@ifpackageloaded{tcolorbox}{}{\usepackage[many]{tcolorbox}}
\makeatother
\makeatletter
\@ifundefined{shadecolor}{\definecolor{shadecolor}{rgb}{.97, .97, .97}}
\makeatother
\makeatletter
\makeatother
\ifLuaTeX
  \usepackage{selnolig}  % disable illegal ligatures
\fi
\IfFileExists{bookmark.sty}{\usepackage{bookmark}}{\usepackage{hyperref}}
\IfFileExists{xurl.sty}{\usepackage{xurl}}{} % add URL line breaks if available
\urlstyle{same} % disable monospaced font for URLs
\hypersetup{
  pdftitle={Draft Proposal - Goldsmiths Research Methods in Psychology},
  pdfauthor={Gordon Wright \& Caroline Rix},
  hidelinks,
  pdfcreator={LaTeX via pandoc}}

\title{Draft Proposal - Goldsmiths Research Methods in Psychology}
\author{Gordon Wright \& Caroline Rix}
\date{2/28/23}

\begin{document}
\frontmatter
\maketitle
\ifdefined\Shaded\renewenvironment{Shaded}{\begin{tcolorbox}[enhanced, sharp corners, frame hidden, boxrule=0pt, borderline west={3pt}{0pt}{shadecolor}, breakable, interior hidden]}{\end{tcolorbox}}\fi

\renewcommand*\contentsname{Table of contents}
{
\setcounter{tocdepth}{2}
\tableofcontents
}
\mainmatter
\bookmarksetup{startatroot}

\hypertarget{proposal-for-psychology-department-research-methods-2024-5}{%
\chapter{Proposal for Psychology Department Research Methods
2024-5}\label{proposal-for-psychology-department-research-methods-2024-5}}

\raggedright

\bookmarksetup{startatroot}

\hypertarget{overview}{%
\chapter{Overview}\label{overview}}

\hypertarget{statistics-and-research-methods-for-psychological-and-behavioural-science}{%
\section{Statistics and Research Methods for Psychological and
Behavioural
Science}\label{statistics-and-research-methods-for-psychological-and-behavioural-science}}

\begin{tcolorbox}[enhanced jigsaw, rightrule=.15mm, opacitybacktitle=0.6, colbacktitle=quarto-callout-warning-color!10!white, breakable, leftrule=.75mm, bottomtitle=1mm, toptitle=1mm, colback=white, titlerule=0mm, opacityback=0, title=\textcolor{quarto-callout-warning-color}{\faExclamationTriangle}\hspace{0.5em}{Warning}, arc=.35mm, bottomrule=.15mm, toprule=.15mm, left=2mm, coltitle=black]

General Goldsmiths bollocks

\end{tcolorbox}

\begin{tcolorbox}[enhanced jigsaw, rightrule=.15mm, opacitybacktitle=0.6, colbacktitle=quarto-callout-important-color!10!white, breakable, leftrule=.75mm, bottomtitle=1mm, toptitle=1mm, colback=white, titlerule=0mm, opacityback=0, title=\textcolor{quarto-callout-important-color}{\faExclamation}\hspace{0.5em}{Important}, arc=.35mm, bottomrule=.15mm, toprule=.15mm, left=2mm, coltitle=black]

The Psychology department at Goldsmiths has a number of differentiating
features, upon which it must build in order to ensure it remains
attractive and distinctive.

\begin{itemize}
\tightlist
\item
  Alchemical, interdisciplinary research programmes and projects
\item
  Passion for empowering students and producing creative, skillful,
  disruptive agents of change
\item
  Blend of Art and Science, craft, entrepreneurship, massively
  transferable mind-set/skill-set
\end{itemize}

Research Methods has to be an adventure, ``Alive'' with enthusiasm,
inspiring curiosity and intellectual problem-solving.

We have to break the A-Level preconception that Research Methods is dry,
maths-like or indeed difficult.

This must be achieved by: - Practical First - The majority of lab
sessions should be practical and `hands on'! - Only present
research/methods/topics that are `ownable' - where we have expertise,
cachet and competitive advantage - and that could NOT be done at school
- `Relevant \& Applicable' above all else. Nothing is taught `because
the BPS says we have to' or that doesn't have widespread use in a range
of future endeavours - Demonstrate and model intellectual virtue,
curiosity, empiricism, skepticism, humility and

\end{tcolorbox}

\begin{tcolorbox}[enhanced jigsaw, rightrule=.15mm, opacitybacktitle=0.6, colbacktitle=quarto-callout-important-color!10!white, breakable, leftrule=.75mm, bottomtitle=1mm, toptitle=1mm, colback=white, titlerule=0mm, opacityback=0, title=\textcolor{quarto-callout-important-color}{\faExclamation}\hspace{0.5em}{Departmental integration}, arc=.35mm, bottomrule=.15mm, toprule=.15mm, left=2mm, coltitle=black]

Integration across the years

Vertical Integration

project delivery

Y0 - MSc

Aggregation of teaching

\end{tcolorbox}

\begin{tcolorbox}[enhanced jigsaw, rightrule=.15mm, opacitybacktitle=0.6, colbacktitle=quarto-callout-note-color!10!white, breakable, leftrule=.75mm, bottomtitle=1mm, toptitle=1mm, colback=white, titlerule=0mm, opacityback=0, title=\textcolor{quarto-callout-note-color}{\faInfo}\hspace{0.5em}{Open Stuff}, arc=.35mm, bottomrule=.15mm, toprule=.15mm, left=2mm, coltitle=black]

We embrace an Open Science approach in our efforts to cultivate your
critical evaluation skills, enhance your understanding of the
significance - and power - of research, and equip you with the necessary
graduate-level skills to collect, handle, and interpret data using
programming software for statistical model development, visualisation
and analysis.

\end{tcolorbox}

\begin{tcolorbox}[enhanced jigsaw, rightrule=.15mm, opacitybacktitle=0.6, colbacktitle=quarto-callout-note-color!10!white, breakable, leftrule=.75mm, bottomtitle=1mm, toptitle=1mm, colback=white, titlerule=0mm, opacityback=0, title=\textcolor{quarto-callout-note-color}{\faInfo}\hspace{0.5em}{Teaching and Assessment}, arc=.35mm, bottomrule=.15mm, toprule=.15mm, left=2mm, coltitle=black]

Through lectures, interactive group discussions, online skills
development modules, and practical lab sessions, we will ignite your
enthusiasm for Psychology and Behavioural Science research and help you
develop the fundamental skills, knowledge - and confidence - required to
become a Psychology literate, disruptive scientist of the future. Tada!

\end{tcolorbox}

AUTHENTIC ASSESSMENT - OPenAI

``Authentic assessment'' generally refers to assignments that reflect
the skills students will use in their post-graduation careers and life
experiences (Archbald, 1991; Gulikers et al., 2004). This is distinct
from traditional assessments such as closed-book exams and academic
essays (Macandrew \& Edwards, 2002). For psychology, authentic
assessments may involve the development of scientific inquiry skills,
such as a practical research report (Halonen et al., 2003). In clinical
psychology, this can include formulation of case studies or mock
practitioner dialogues (Villarroel et al., 2018). Other examples of
authentic assessments include writing letters to non-psychology
audiences (Cranney, 2013), critical evaluation of media materials
(Halpern \& Butler, 2011), written reflections of volunteering
(Hadlington; as cited in Taylor \& Hulme, 2015), and research project
interviews (Turner \& Davila-Ross, 2015). All of these are designed to
promote psychological literacy by encouraging communication with
non-experts and application of psychology content (Hulme, 2014).

Central to psychological literacy is the effective communication of
psychology in practice (Trapp et al., 2011). To this end, blog writing
may be an effective assessment for capturing and promoting students'
psychological literacy. Blogs are a web-based form of journal or writing
platform, and are useful for disseminating information (Richardson,
2006). They are also increasingly popular for assessing written language
skills (Kim, 2008; Lee, 2010; Raith, 2009; Williams \& Jacobs, 2004).
Blog writing may help students take different perspectives, develop a
critical appraisal of content, and become more self-reflective (Frye et
al., 2010; Jimoyiannis \& Angelaina, 2012). It also empowers and engages
students (Creme \& Hunt, 2002), a core goal of psychology education
(Ishak \& Salter, 2017). It enhances academic language learning (Murray
et al., 2007) and assesses different skills than traditional forms of
assessment (Morris et al., 2013). Studies show that students report high
levels of engagement with blog writing (Hindley, 2018).

Blog writing may help students overcome the challenge of academic essay
writing (Ishak \& Salter, 2017) since it is less rigidly governed by
academic norms (Bennett et al., 2012) and involves more reflective,
creative, and critical tone (Arslan \& Aysel, 2010; Novakovich \& Long,
2013). It also provides an opportunity to communicate psychology content
to a wider audience (Gardiner et al., 2018; Schmidt, 2008; Relojo,
2017). It has been embedded successfully in many different subjects in
higher education, such as pharmacy (Dunne \& Ryan, 2016), anthropology
(Walker \& Chatzigavriil, 2017), professional development (Shanks,
2020), and English literature (Agarwal, 2017). The British Psychological
Society, 2019 and the American Psychological Association (2013) stress
the importance of critical evaluation and reflection, which is
compatible with blog writing. Research shows that blog writing prompts
critical thinking and reflection (Chretien et al., 2008; Novakovich,
2016). Challenges to implementing blog writing as an assessment include
students having little to no previous experience (Kerawalla et al.,
2009). Blog writing should be grounded effectively in course materials,
have clear expectations, and be understood through a lens of
psychological literacy (Cranney et al., 2008, 2012). We propose that
blog writing may provide a useful opportunity to allow students to apply
their psychology content in a critical, creative, and non-conventional
way.

\begin{itemize}
\tightlist
\item
  OPENAIabove
\end{itemize}

BPS Guidelines from OpenAI

1. Ensure that students have a clear understanding of the research
methods used in their field of study.

\begin{enumerate}
\def\labelenumi{\arabic{enumi}.}
\setcounter{enumi}{1}
\item
  Provide opportunities for students to develop critical thinking skills
  in relation to the research methods used.
\item
  Encourage students to practice the research methods in a safe and
  secure environment.
\item
  Develop a range of teaching materials and resources that best support
  student learning.
\item
  Encourage students to become independent learners by providing
  appropriate scaffolding and support.
\item
  Encourage students to become active participants in their learning
  process.
\item
  Facilitate meaningful discussions about research methods and the
  application of research methods in practice.
\item
  Incorporate ethical considerations into the research methods teaching.
\item
  Promote the use of technology to support the teaching of research
  methods.
\item
  Monitor and evaluate students' progress throughout the teaching
  period.
\end{enumerate}

Source: British Psychological Society (BPS). (2020). Guidance for the
teaching of undergraduate research methods. Retrieved from
https://www.bps.org.uk/publications/guidance-teaching-undergraduate-research-methods

\begin{tcolorbox}[enhanced jigsaw, rightrule=.15mm, opacitybacktitle=0.6, colbacktitle=quarto-callout-note-color!10!white, breakable, leftrule=.75mm, bottomtitle=1mm, toptitle=1mm, colback=white, titlerule=0mm, opacityback=0, title=\textcolor{quarto-callout-note-color}{\faInfo}\hspace{0.5em}{BPS website}, arc=.35mm, bottomrule=.15mm, toprule=.15mm, left=2mm, coltitle=black]

Research methods Research methods must be delivered at Level 5 or Level
6. Research methods are integral to Psychology and students obtain a
sound knowledge of, and a proven ability to use, a range of methods
appropriately. Knowledge and understanding of how to obtain and analyse
evidence is best acquired and demonstrated through extensive and
progressive empirical work in laboratory and naturalistic settings
through all stages of a degree. {[}section 3.4 Subject Benchmark
Statement{]} Psychology students learn the basic principles of sound
data collection. Given the broad theoretical scope of Psychology,
rigorous specialist training is required to engender a critical
understanding of the role of experimental design, the choice of research
methods employed, and the analytic approach taken, for testing
psychological theories. {[}section 3.5 Subject Benchmark Statement{]}
Programmes' coverage of research methods should support students'
engagement with the sub-areas listed above, and should be directed
towards supporting students' attainment of the subject-specific skills
highlighted in section 4.4 of the Subject Benchmark Statement. As they
progress through the programme, students should be able to: • apply
multiple perspectives to psychological issues, recognising that
psychology involves a range of research methods, theories, evidence and
applications; • integrate ideas and findings across the multiple
perspectives in psychology and recognise distinctive psychological
approaches to relevant issues; • identify and evaluate patterns in
behaviour, psychological functioning and experience; • generate and
explore hypotheses and research questions drawing on relevant theory and
research; • carry out empirical studies involving a variety of methods
of data collection, including experiments, observation, questionnaires,
interviews and focus groups; • analyse, present and evaluate
quantitative and qualitative data and evaluate research findings; •
employ evidence-based reasoning and examine practical, theoretical and
ethical issues associated with the range of methodologies; • use a
variety of psychological tools, including specialist software,
laboratory equipment and psychometric instruments; • apply psychological
knowledge ethically and safely to real world problems; and • critically
evaluate psychological theory and research.

\end{tcolorbox}

Communicate complex information effectively using appropriate written,
oral, graphical and electronic means, taking into account diversity
among individuals to whom the information is communicated.

Explain the potential impact of psychological research and theory on a
broad range of real world settings and situations (e.g., classrooms,
industry, commerce, healthcare, as well as local and global
communities).

Problem-solve and reason scientifically. Specifically, graduates will be
able to identify and pose research questions, consider alternative
approaches to their solutions, and evaluate outcomes.

Be sensitive to contextual and interpersonal factors. Graduates will be
familiar with the complexity of the factors that shape behaviour and
social interaction which, in turn, will make them more aware of the
bases of problems and interpersonal conflicts.

or Be a self-critical learner, showing sensitivity to contextual and
interpersonal factors. Graduates will be familiar with the complexity of
the factors that shape behaviour and social interaction which, in turn,
will make them more aware of the bases of problems and interpersonal
conflicts.

Show an understanding of various research paradigms, methods, and
evaluation procedures, including statistical analysis, as well as their
constraints.

Design, carry out, evaluate and interpret scientifically rigorous and
ethically sound studies both independently and collaboratively,
utilizing quantitative and qualitative methods, statistical analysis and
modern digital software.

Psychological literacy is the ability to understand and apply
psychological principles and theories to everyday life. This includes
the ability to understand how psychological processes and phenomena
influence our behavior, emotions, thoughts, and relationships. It also
includes the capacity to use psychological knowledge to make informed
decisions and to better understand, explain, and predict the behavior of
self and others.

Psychology graduates are highly sought after by employers due to their
ability to formulate and communicate well-reasoned, evidence-based, and
statistically defensible arguments based on their expertise in the study
of human behavior and its causes. On top of this, psychology graduates
possess the skills to work independently or collaboratively, as well as
strong numerical capabilities, verbal and written communication skills,
and an up-to-date knowledge of digital technologies applicable to a wide
range of occupational fields.

\hypertarget{intended-learning-outcomes}{%
\subsection{Intended Learning
Outcomes}\label{intended-learning-outcomes}}

Intended Learning Outcomes

Create reproducible data analysis scripts and reports within the R
statistical programming environment.

QAA Benchmarks

Subject Knowledge and Understanding

6.3.4 demonstrate detailed knowledge of several specialised areas and/or
applications, some of which are at the cutting edge of research in the
discipline 6.3.5 demonstrate a systematic knowledge of a range of
research paradigms, research methods and measurement techniques,
including statistics and probability, and be aware of their limitations.

Subject-specific skills

PS510XX - RM1 - Introduction to Research Methods and Data Skills

PS520XX - RM2 - Research Methods in Practice and Data Skills

PS530XX - RM3 - Research Project Incubator

*PS710XX - Practical Research Skills

Lectures - Overview of key concepts/context and preview Lab practicals /
Data Skills

Labs - Practical or activity based (inc. Group Work)

\hypertarget{executive-summary-by-year}{%
\subsection{Executive Summary by year}\label{executive-summary-by-year}}

Y1 - showcase and active participation/skill development

Y2 - Practical drive towards self-motivated research

Y3 - Competent research

Social Constructivist

PeerMark

Podcast/Webpage/Blog

Integrate own interest/guided by stream/lab

\hypertarget{technical-overview}{%
\subsection{Technical Overview}\label{technical-overview}}

\begin{figure}

\begin{minipage}[t]{0.20\linewidth}

{\centering 

\raisebox{-\height}{

\includegraphics[width=0.67708in,height=\textheight]{images/R_logo.png}

}

\caption{R language}

}

\end{minipage}%
%
\begin{minipage}[t]{0.15\linewidth}

{\centering 

\raisebox{-\height}{

\includegraphics[width=0.52083in,height=\textheight]{images/posit_logo.png}

}

\caption{Posit (Formerly RStudio)}

}

\end{minipage}%
%
\begin{minipage}[t]{0.64\linewidth}

{\centering 

\raisebox{-\height}{

\includegraphics[width=2.16667in,height=\textheight]{images/quarto_logo.png}

}

\caption{Quarto Academic Writing Suite}

}

\end{minipage}%

\end{figure}

\begin{figure}

\begin{minipage}[t]{0.50\linewidth}

{\centering 

\hypertarget{list-one}{%
\subsection{List One}\label{list-one}}

\begin{itemize}
\tightlist
\item
  Item A
\item
  Item B
\item
  Item C
\end{itemize}

}

\end{minipage}%
%
\begin{minipage}[t]{0.50\linewidth}

{\centering 

\hypertarget{list-two}{%
\subsection{List Two}\label{list-two}}

\begin{itemize}
\tightlist
\item
  Item X
\item
  Item Y
\item
  Item Z
\end{itemize}

}

\end{minipage}%

\end{figure}

R will be used. Gold standard statistical programming language

For literate programming (The concept of
\href{https://en.wikipedia.org/wiki/Literate_programming}{\textbf{``literate
programming''}} was originally introduced by
\href{http://www.literateprogramming.com/knuthweb.pdf}{Donald Knuth} in
1984 )

Formerly RStudio. The Interactive Development Enviornment for use of R.

\hypertarget{hours-specification-e.g.-years-1-2}{%
\subsection{Hours specification (e.g.) Years 1 \&
2}\label{hours-specification-e.g.-years-1-2}}

\begin{longtable}[]{@{}lll@{}}
\caption{Notional Hours}\tabularnewline
\toprule()
Activity & Time & Note \\
\midrule()
\endfirsthead
\toprule()
Activity & Time & Note \\
\midrule()
\endhead
Lectures & 40 & 2hrs/week \\
Labs & 40 & 2hrs/week \\
Data Skills (Online) & 40 & 2hrs/week \\
Guided Reading/viewing & 40 & 2hrs/week \\
RPS & 20 & 1hr/week \\
Independent Study/Coursework & 120 & 6hr/week \\
\bottomrule()
\end{longtable}

\hypertarget{programme-overview}{%
\section{Programme Overview}\label{programme-overview}}

\hypertarget{pre-arrival-onwards-onboarding}{%
\subsection{Pre-Arrival onwards /
Onboarding}\label{pre-arrival-onwards-onboarding}}

Showcase in Induction week - Staff labs and research projects for the
year.

Year One students self-test

MSc Students - ditto and ability to shop around for supervision

Year 2 develop their pods? Show Y1 and Foundations what they did last
year

Year 3/MSc students - Research Bootcamp and refreshers/skills workshops

Support PhD students and staff

\hypertarget{shock-and-awe---shatter-the-a-level-preconceptions}{%
\subsection{Shock and Awe - Shatter the A-Level
preconceptions}\label{shock-and-awe---shatter-the-a-level-preconceptions}}

\hypertarget{vertically-integrated-projects-via-labs}{%
\subsection{Vertically Integrated Projects via
`Labs'}\label{vertically-integrated-projects-via-labs}}

\hypertarget{heartdata-week-recruitment-forward-prep}{%
\subsection{HeartData week (recruitment \& forward
prep)}\label{heartdata-week-recruitment-forward-prep}}

Potentially Reading Week Term 2? Or week before/after?

Allows all levels of students to blitz data and to showcase their work
for external stakeholders and to make a department-wide event.

\bookmarksetup{startatroot}

\hypertarget{stuff}{%
\chapter{STUFF}\label{stuff}}

\hypertarget{personal-development-skills}{%
\section{\texorpdfstring{\textbf{Personal development
skills}}{Personal development skills}}\label{personal-development-skills}}

\begin{itemize}
\item
  self-management
\item
  team working
\item
  problem solving
\item
  application of information skills
\item
  communication
\item
  application of numeracy skills
\item
  specialist skills
\end{itemize}

OPENAI of below

McGovern et al.~(2010) define psychological literacy as `being
insightful and reflective about one's own and others' behaviour and
mental processes' (p.11). This concept captures the ability of a
psychology student to apply the knowledge and skills gained during their
education to all aspects of life, such as the workplace, personal life,
and wider social context. Cranney et al.~(2012, p.4) adopt a similar
definition of psychological literacy as `the general capacity to
adaptively and intentionally apply psychology to meet personal,
professional and societal needs'. Dunn (2009) further explains that
psychologically literate people are able to use psychology to solve
issues on a local, civic, and national level by looking to data instead
of personal opinion.

In order to promote psychological literacy, Dunn et al.~(2011) and Mair
et al.~(2013) suggest that the applications of psychology should be
recognised and taught in a real-world context; Biggs (1996) recommends
that courses should be constructively aligned to explicitly include
psychological literacy in the learning outcomes and assessments; and
Akhurst et al.~(in press), Bernstein (2011), Cranney and Dunn (2011),
McGovern (2011) and Zinkiewicz et al.~(2003) suggest that psychological
literacy should be modelled in professional lives through interactions
with colleagues and students, and by using psychology to inform teaching
practices, solve problems, and ensure inclusivity.

\begin{itemize}
\item
  \begin{tcolorbox}[enhanced jigsaw, rightrule=.15mm, opacitybacktitle=0.6, colbacktitle=quarto-callout-note-color!10!white, breakable, leftrule=.75mm, bottomtitle=1mm, toptitle=1mm, colback=white, titlerule=0mm, opacityback=0, title=\textcolor{quarto-callout-note-color}{\faInfo}\hspace{0.5em}{Note}, arc=.35mm, bottomrule=.15mm, toprule=.15mm, left=2mm, coltitle=black]

  \begin{itemize}
  \item
    McGovern et al.~(2010) defines psychological literacy as `being
    insightful and reflective about one's own and others behaviour and
    mental processes' (p.11). It is, in essence, the general capacity to
    apply psychological knowledge to real-world scenarios.
  \item
    Dunn (2009) discusses this, claiming that `psychologically literate
    people can use what they know about psychology to solve home-based,
    local, civic, and even national matters by looking to data instead
    of personal opinion'.
  \item
    McGovern et al.~(2010, p.11) define psychological literacy as `being
    insightful and reflective about one's own and others' behaviour and
    mental processes' and having the ability to apply `psychological
    principles to personal, social, and organizational issues in work,
    relationships and the broader community'. Cranney et al.~(2012, p.4)
    adopt a similar stance, describing psychological literacy as `the
    general capacity to adaptively and intentionally apply psychology to
    meet personal, professional and societal needs'. The concept of
    psychological literacy thus captures the ability of a psychology
    student to apply the knowledge and skills that they acquire during
    their education to all aspects of life: the workplace, their
    personal lives and the wider social context. As such, psychological
    literacy may provide a lens through which we can view the wider
    benefits of psychology education. It may also hold out some hope to
    students who may be coming to accept that their aspirations towards
    a career in psychology are likely to remain unfulfilled, but who
    remain enthused and inspired by the subject, and are looking for
    ways to continue to stay in touch with it.
  \item
    We need to:\\
    I~~ ~recognise and to teach the applications of psychology, its
    relevance to the real world and the transferability of skills,
    rather than always teaching it in a theoretical context (Dunn et
    al., 2011; Mair et al., 2013);\\
    I~~ ~constructively align (Biggs, 1996) our courses to explicitly
    include psychological literacy in our learning outcomes, our
    teaching, and the assessments that we give to our students (Dunn et
    al., 2011; Trapp, 2010; Trapp et al., 2011);\\
    I~~ ~model psychological literacy in our own professional lives,
    through our interactions with colleagues and students, using
    psychology to inform our teaching practices, solve problems and
    ensure inclusivity (Akhurst et al., in press; Bernstein, 2011;
    Cranney and Dunn, 2011; McGovern, 2011; Zinkiewicz et al., 2003).
  \item
    \emph{Psychological Literacy}

    Barnett, R. (2006). Graduate attributes in an age of uncertainty.
    \emph{Graduate attributes, learning and employability}, 49-65.

    Cranney, J., Botwood, L., \& Morris, S. (2012). National standards
    for psychological literacy and global citizenship: Outcomes of
    undergraduate psychology education. \emph{Australia: The University
    of New South Wales.}

    Cranney, J., Morris, S., Martin, F. H., Provost, S., Zinkiewicz, L.,
    Reece, J \& Earl, J. (2011). Psychological literacy and applied
    psychology in undergraduate education.

    Dunn (2009) Psychology Today.
    \href{https://www.psychologytoday.com/blog/head-the-class/200909/thinking-about-psychological-literacy}{\textbf{https://www.psychologytoday.com/blog/head-the-class/200909/thinking-about-psychological-literacy}}

    Haigh, M., \& Clifford, V. A. (2010). Widening the graduate
    attribute debate: a Higher education for Global citizenship.
    \emph{Brookes eJournal of Learning and Teaching}, \emph{2}(5).

    McGovern T. V., Corey L., Cranney J., Dixon W., Holmes J. D.,
    Kuebli, et al.~. (2010). Psychologically literate citizens, in
    Undergraduate Education in Pscyhology: A Blueprint for the Future of
    the Discipline, ed Halpern D. F., editor. (Washington, DC: American
    Psychological Association; ), 9--27.

    Murdoch, D. D. (2016). Psychological literacy: proceed with caution,
    construction ahead. \emph{Psychology research and behavior
    management}, \emph{9}, 189.

    Trapp, A., Banister, P., Ellis, J., Latto, R., Miell, D., \& Upton,
    D. (2011). The future of undergraduate psychology in the United
    Kingdom. \emph{Higher Education Academy Psychology Network,
    University of York, York}.
  \item
    Akhurst, J. (2013). Enhancing psychology students' employability
    through international community-based work placements. Higher
    Education Academy. Retrieved 30 September 2014 from
    \href{http://www.tinyurl.com/pd6bmtg}{\textbf{tinyurl.com/pd6bmtg}}\\
    Akhurst, J., Coxon, M. \& Hulme, J. (in press). Applying psychology
    to psychology learning and teaching. York: Higher Education
    Academy.\\
    Barnett, R. (2010). Life-wide education: A new and transformative
    concept for higher education? In N. Jackson \& R. Law (Eds.)
    Enabling a More Complete Education {[}Conference proceedings{]}.
    University of Surrey. (Available at
    \href{http://www.http//lifewidelearningconference\%20.pbworks.com/E-proceedings}{\textbf{http://lifewidelearningconferencepbworks.com/E-proceedings}})\\
    Bernstein, D. (2011). A scientist-educator perspective on
    psychological literacy. In J. Cranney \& D. Dunn (Eds.) The
    psychologically literate citizen. New York: Oxford University
    Press.\\
    Biggs, J. (1996). Enhancing teaching through constructive alignment.
    Higher Education, 32(3), 347--364.\\
    Boneau, C.A. (1990). Psychological literacy: A first approximation.
    American Psychologist, 45, 891--900.\\
    Bromnick, R. \& Horowitz, A. (2013). Reframing employability:
    Exploring career-related values in psychology undergraduates. Paper
    presented at the HEA STEM Annual Learning and Teaching Conference,
    University of Birmingham, April. Retrieved 30 September 2014 from
    \href{http://www.tinyurl.com/mm9xojq}{\textbf{tinyurl.com/mm9xojq}}\\
    CBI/NUS (2011). Working towards your future: Making the most of your
    time in higher education. London: CBI. Retrieved 22 July 2014 from
    \href{http://www.tinyurl.com/lsqsdwu}{\textbf{tinyurl.com/lsqsdwu}}\\
    CIPD (2011). The coaching climate. London: CIPD. (Available via
    \href{http://www.www.cipd.co.uk/hr-resources/survey-reports/coaching-climate-2011.aspx}{\textbf{www.cipd.co.uk/hr-resources/survey-reports/coaching-climate-2011.aspx}})\\
    Cranney, J., Botwood, L. \& Morris, S. (2012). National standards
    for psychological literacy and global citizenship: Outcomes of
    undergraduate psychology education. Sydney, NSW: Office for Learning
    and Teaching. Retrieved 22 July 2014 from
    \href{http://www.tinyurl.com/q98zg4y}{\textbf{tinyurl.com/q98zg4y}}\\
    Cranney, J. \& Dunn, D. (2011). What the world needs now is
    psychological literacy. In J. Cranney \& D. Dunn (Eds.) The
    psychologically literate citizen. New York: Oxford University
    Press.\\
    Dunn, D., Cautin, R.L. \& Gurung, A.R. (2011). Curriculum matters:
    Structure, content, and psychological literacy. In J. Cranney \& D.
    Dunn (Eds.) The psychologically literate citizen. New York: Oxford
    University Press.\\
    Grabinger, R.S. \& Dunlap, J.C. (1995). Rich environments for active
    learning: A definition. Research in Learning Technology, 3(2),
    5--34.\\
    Halpern, D. (2010). Undergraduate education in psychology: A
    blueprint for the future of the discipline. Washington, DC: APA.\\
    Harkness, F. (2013). He's just not that into you. The Psychologist,
    26(5), 314--315.\\
    Harnish, R. \& Bridges, K.R. (2012). Promoting student engagement:
    Using community service-learning projects in undergraduate
    psychology. PRISM: A Journal of Regional Engagement, 1(2). Retrieved
    22 July 2014 from
    \href{http://www.http//encompass.eku.edu/prism/vol1/iss2/1\%E2\%88\%9A}{\textbf{http://encompass.eku.edu/prism/vol1/iss2/1√}}

    \hfill\break
  \item
    \href{https://thepsychologist.bps.org.uk/volume-27/december-2014/psychological-literacy-classroom-real-world}{\textbf{psychological
    literacy}}
  \end{itemize}

  \end{tcolorbox}
\item
\end{itemize}

\begin{figure}

\includegraphics[width=0.36458in,height=\textheight]{images/GitHub-Mark.png} \hfill{}

\end{figure}

\href{http://creativecommons.org/licenses/by/4.0/}{\includegraphics{images/creative_commons.png}}
This website/book is licensed under a
\href{http://creativecommons.org/licenses/by/4.0/}{Creative Commons
Attribution 4.0 International License.}

\bookmarksetup{startatroot}

\hypertarget{pedagogical-delivery-overview}{%
\chapter{Pedagogical \& Delivery
Overview}\label{pedagogical-delivery-overview}}

\bookmarksetup{startatroot}

\hypertarget{general-fun}{%
\chapter{General FUN}\label{general-fun}}

Practical first.

Flipped stats

Evidence suggests that a flipped approach to teaching statistics
significantly improves student performance (across a range of metrics)
at the mean and throughout the achievement distribution, while
controlling for baseline student characteristics Sathy \& Moore (2020)
(see tables 13.3 \& 13.4 - specifically Anxiety, preparedness and
under-represented minority status).

\hypertarget{assessments}{%
\section{Assessments}\label{assessments}}

\begin{tcolorbox}[enhanced jigsaw, rightrule=.15mm, opacitybacktitle=0.6, colbacktitle=quarto-callout-note-color!10!white, breakable, leftrule=.75mm, bottomtitle=1mm, toptitle=1mm, colback=white, titlerule=0mm, opacityback=0, title=\textcolor{quarto-callout-note-color}{\faInfo}\hspace{0.5em}{assessment types}, arc=.35mm, bottomrule=.15mm, toprule=.15mm, left=2mm, coltitle=black]

Blog

Podcast

Website

Posters

Information Packs Macandrew \& Edwards (2002)

Portfolios (or ProcessFolio)

Interview (and self-reflection)

Clinical Assessment

Skills assessment

Oral Examination (mock job interview)

Self and Peer evaluation

Literature Review (with or without annotations)

Annotated Code

Computational documents (Lab Reports)

Design and Proposal (Pre-registration)

Data analysis (Results section)

\end{tcolorbox}

``Authentic assessment'' (Archbald, 1991) refers to assignments that are
designed to reflect the skills that students will practice in their
careers and other life experiences after graduation
(\href{https://journals-sagepub-com.gold.idm.oclc.org/doi/full/10.1177/00986283211027278\#bibr24-00986283211027278}{Gulikers
et al., 2004}), looking beyond traditional assessment such as
closed-book exams and traditional formats of academic essays
(\href{https://journals-sagepub-com.gold.idm.oclc.org/doi/full/10.1177/00986283211027278\#bibr44-00986283211027278}{Macandrew
\& Edwards, 2002}). The impact of employing Authentic Assessments range
from increased engagement and satisfaction through to heightened
employability skills (Sokhanvar et al., 2021)

\hypertarget{projects}{%
\section{Projects}\label{projects}}

``Authentic assessment'' refers to assignments that are designed to
reflect the skills that students will practice in their careers and
other life experiences after graduation
(\href{https://journals-sagepub-com.gold.idm.oclc.org/doi/full/10.1177/00986283211027278\#bibr4-00986283211027278}{Archbald,
1991};
\href{https://journals-sagepub-com.gold.idm.oclc.org/doi/full/10.1177/00986283211027278\#bibr24-00986283211027278}{Gulikers
et al., 2004}), looking beyond traditional assessment such as
closed-book exams and traditional formats of academic essays
(\href{https://journals-sagepub-com.gold.idm.oclc.org/doi/full/10.1177/00986283211027278\#bibr44-00986283211027278}{Macandrew
\& Edwards, 2002}). In general psychology, for example, authentic
assessments may include the explicit development of scientific inquiry
skills, such as a practical research report
(\href{https://journals-sagepub-com.gold.idm.oclc.org/doi/full/10.1177/00986283211027278\#bibr25-00986283211027278}{Halonen
et al., 2003}). Or, for more applied facets such as clinical psychology,
this may also include formulation of case studies or engagement in mock
practitioner dialogues
(\href{https://journals-sagepub-com.gold.idm.oclc.org/doi/full/10.1177/00986283211027278\#bibr66-00986283211027278}{Villarroel
et al., 2018}). Previous examples of authentic assessments designed to
foster and assess psychological literacy include asking students to
write letters to non-psychology audiences
(\href{https://journals-sagepub-com.gold.idm.oclc.org/doi/full/10.1177/00986283211027278\#bibr14-00986283211027278}{Cranney,
2013}), critical evaluation of media materials
(\href{https://journals-sagepub-com.gold.idm.oclc.org/doi/full/10.1177/00986283211027278\#bibr26-00986283211027278}{Halpern
\& Butler, 2011}), written reflections of volunteering (Hadlington; as
cited in
\href{https://journals-sagepub-com.gold.idm.oclc.org/doi/full/10.1177/00986283211027278\#bibr63-00986283211027278}{Taylor
\& Hulme, 2015}), and research project interviews
(\href{https://journals-sagepub-com.gold.idm.oclc.org/doi/full/10.1177/00986283211027278\#bibr65-00986283211027278}{Turner
\& Davila-Ross, 2015}). All of these examples are designed to encourage
a psychologically literate approach to psychology content because they
encourage communication with non-experts and application of psychology
content
(\href{https://journals-sagepub-com.gold.idm.oclc.org/doi/full/10.1177/00986283211027278\#bibr33-00986283211027278}{Hulme,
2014}).

Indeed, central to the notion of psychological literacy is the effective
communication of psychology in practice
(\href{https://journals-sagepub-com.gold.idm.oclc.org/doi/full/10.1177/00986283211027278\#bibr64-00986283211027278}{Trapp
et al., 2011}). Therefore, a psychologically literate education should
encourage students to communicate their knowledge in clear and
accessible ways. With this in mind, here, we propose that blog writing
may also be an effective assessment for capturing and promoting
students' psychological literacy. Blog writing inherently aligns with
the agenda of psychological literacy because it aims to communicate
often complex psychology knowledge in an accessible way
(\href{https://journals-sagepub-com.gold.idm.oclc.org/doi/full/10.1177/00986283211027278\#bibr38-00986283211027278}{Jolley
et al., 2016};
\href{https://journals-sagepub-com.gold.idm.oclc.org/doi/full/10.1177/00986283211027278\#bibr43-00986283211027278}{Lin
et al., 2007}).

Primary Citation Pownall et al. (2023) Blogs, which are a web-based form
of journal or writing platform, are a powerful tool in the wide
dissemination of information in the modern media mix - relevant to
current students
(\href{https://journals-sagepub-com.gold.idm.oclc.org/doi/full/10.1177/00986283211027278\#bibr56-00986283211027278}{Richardson,
2006}) .

Blog writing is thought to offer the opportunity to take different
perspectives in writing, develop a more critical appraisal of the
content, and thus can prompt higher levels of self-reflection (e.g.,
\href{https://journals-sagepub-com.gold.idm.oclc.org/doi/full/10.1177/00986283211027278\#bibr22-00986283211027278}{Frye
et al., 2010};
\href{https://journals-sagepub-com.gold.idm.oclc.org/doi/full/10.1177/00986283211027278\#bibr37-00986283211027278}{Jimoyiannis
\& Angelaina, 2012}). Activities that encourage creative participation
in the process of academic writing have been found to empower and engage
students
(\href{https://journals-sagepub-com.gold.idm.oclc.org/doi/full/10.1177/00986283211027278\#bibr18-00986283211027278}{Creme
\& Hunt, 2002}), a core goal of a psychology education
(\href{https://journals-sagepub-com.gold.idm.oclc.org/doi/full/10.1177/00986283211027278\#bibr36-00986283211027278}{Ishak
\& Salter, 2017}). Therefore, blog writing promotes wider academic
language learning
(\href{https://journals-sagepub-com.gold.idm.oclc.org/doi/full/10.1177/00986283211027278\#bibr49-00986283211027278}{Murray
et al., 2007}) and thus complements more traditional forms of
assessments by allowing a different set of skills to be assessed
(\href{https://journals-sagepub-com.gold.idm.oclc.org/doi/full/10.1177/00986283211027278\#bibr47-00986283211027278}{Morris
et al., 2013}). Across the literature, studies show that students report
high levels of engagement with blog writing exercises, owing to the high
levels of creativity associated with this form of writing
(\href{https://journals-sagepub-com.gold.idm.oclc.org/doi/full/10.1177/00986283211027278\#bibr31-00986283211027278}{Hindley,
2018}).

Research demonstrates students often report feeling unprepared and
overwhelmed at the perceived ``rules'' of academic essay writing
(\href{https://journals-sagepub-com.gold.idm.oclc.org/doi/full/10.1177/00986283211027278\#bibr36-00986283211027278}{Ishak
\& Salter, 2017}). Blog writing should help students to overcome this
pedagogic challenge, given that blogs are governed less rigidly by
academic norms
(\href{https://journals-sagepub-com.gold.idm.oclc.org/doi/full/10.1177/00986283211027278\#bibr6-00986283211027278}{Bennett
et al., 2012}) and typically involve more reflective, creative, and
critical tone
(\href{https://journals-sagepub-com.gold.idm.oclc.org/doi/full/10.1177/00986283211027278\#bibr5-00986283211027278}{Arslan
\& Aysel, 2010};
\href{https://journals-sagepub-com.gold.idm.oclc.org/doi/full/10.1177/00986283211027278\#bibr52-00986283211027278}{Novakovich
\& Long, 2013}). Writing in a clear and accessible way is an important
transferable skill and undergraduate outcomes across subjects in higher
education
(\href{https://journals-sagepub-com.gold.idm.oclc.org/doi/full/10.1177/00986283211027278\#bibr29-00986283211027278}{Hawkey
\& Barker, 2004};
\href{https://journals-sagepub-com.gold.idm.oclc.org/doi/full/10.1177/00986283211027278\#bibr42-00986283211027278}{Leki
\& Carson, 1994}). Blog writing can, in theory, enhance these skills,
given its highly reflective nature and deviance away from the
conventions of academic writing with which some students struggle
(\href{https://journals-sagepub-com.gold.idm.oclc.org/doi/full/10.1177/00986283211027278\#bibr19-00986283211027278}{Dippold,
2009};
\href{https://journals-sagepub-com.gold.idm.oclc.org/doi/full/10.1177/00986283211027278\#bibr59-00986283211027278}{Soysa
et al., 2013};
\href{https://journals-sagepub-com.gold.idm.oclc.org/doi/full/10.1177/00986283211027278\#bibr69-00986283211027278}{Xie
et al., 2008}). Similarly, blog writing also provides a useful
opportunity to communicate psychology content to a wider audience.
Scholars have noted how traditional forms of academic psychology
dissemination, such as journal articles, are typically inaccessible to a
general audience, both in terms of language use and access (i.e.,
because they are published behind a paywall that requires institutional
subscriptions;
\href{https://journals-sagepub-com.gold.idm.oclc.org/doi/full/10.1177/00986283211027278\#bibr55-00986283211027278}{Relojo,
2017}). Blog writing may thus be an important platform to mediate the
relationship between science and the general public (e.g.,
\href{https://journals-sagepub-com.gold.idm.oclc.org/doi/full/10.1177/00986283211027278\#bibr23-00986283211027278}{Gardiner
et al., 2018};
\href{https://journals-sagepub-com.gold.idm.oclc.org/doi/full/10.1177/00986283211027278\#bibr57-00986283211027278}{Schmidt,
2008}), which again is a facet of the psychological literacy approach
(\href{https://journals-sagepub-com.gold.idm.oclc.org/doi/full/10.1177/00986283211027278\#bibr33-00986283211027278}{Hulme,
2014}).

Blog writing as an assessment has been embedded successfully in many
different academic subjects in higher education, such as in pharmacy
(\href{https://journals-sagepub-com.gold.idm.oclc.org/doi/full/10.1177/00986283211027278\#bibr20-00986283211027278}{Dunne
\& Ryan, 2016}), anthropology
(\href{https://journals-sagepub-com.gold.idm.oclc.org/doi/full/10.1177/00986283211027278\#bibr67-00986283211027278}{Walker
\& Chatzigavriil, 2017}), professional development
(\href{https://journals-sagepub-com.gold.idm.oclc.org/doi/full/10.1177/00986283211027278\#bibr58-00986283211027278}{Shanks,
2020}), and English literature
(\href{https://journals-sagepub-com.gold.idm.oclc.org/doi/full/10.1177/00986283211027278\#bibr1-00986283211027278}{Agarwal,
2017}). Indeed, given that the
\href{https://journals-sagepub-com.gold.idm.oclc.org/doi/full/10.1177/00986283211027278\#bibr8-00986283211027278}{British
Psychological Society, 2019} note in their degree accreditation
standards that ``critical evaluation and reflection'' is a key graduate
attribute for psychology programs, blog writing may be particularly
compatible with psychology undergraduate content. The blog approach to
writing has been promoted through BPS-led initiatives, such as the
Voices in Psychology program
(\href{https://journals-sagepub-com.gold.idm.oclc.org/doi/full/10.1177/00986283211027278\#bibr60-00986283211027278}{Sutton
\& Pownall, 2018},
\href{https://journals-sagepub-com.gold.idm.oclc.org/doi/full/10.1177/00986283211027278\#bibr61-00986283211027278}{2019}).
This also extends to the American Psychological Association's guidelines
for undergraduate psychology majors
(\href{https://journals-sagepub-com.gold.idm.oclc.org/doi/full/10.1177/00986283211027278\#bibr3-00986283211027278}{2013}),
who too stress that scientific inquiry, critical thinking, and
communication are among the core goals of a psychology degree.

Moreover, research shows that blog writing prompts critical thinking and
reflection
(\href{https://journals-sagepub-com.gold.idm.oclc.org/doi/full/10.1177/00986283211027278\#bibr12-00986283211027278}{Chretien
et al., 2008};
\href{https://journals-sagepub-com.gold.idm.oclc.org/doi/full/10.1177/00986283211027278\#bibr51-00986283211027278}{Novakovich,
2016}). For example,
\href{https://journals-sagepub-com.gold.idm.oclc.org/doi/full/10.1177/00986283211027278\#bibr51-00986283211027278}{Novakovich
(2016)} investigated whether there are differences in the quality of
writing generated through in-class workshops between the use of
traditional methods compared with blog writing. They concluded that blog
writing fostered more complex literacy skills, which is echoed by other
empirical studies (e.g.,
\href{https://journals-sagepub-com.gold.idm.oclc.org/doi/full/10.1177/00986283211027278\#bibr2-00986283211027278}{Alsamadani,
2018};
\href{https://journals-sagepub-com.gold.idm.oclc.org/doi/full/10.1177/00986283211027278\#bibr21-00986283211027278}{Febianti
\& Wahyuni, 2019}).

However, there are also some challenges in implementing blogs as an
effective assessment format. For example, students often have little to
no previous experience of blogging which can impede engagement with it
as an assessment
(\href{https://journals-sagepub-com.gold.idm.oclc.org/doi/full/10.1177/00986283211027278\#bibr39-00986283211027278}{Kerawalla
et al., 2009}). Similarly,
\href{https://journals-sagepub-com.gold.idm.oclc.org/doi/full/10.1177/00986283211027278\#bibr39-00986283211027278}{Kerawalla
et al. (2009)} warned that blog writing as an assessment in higher
education should be grounded effectively in the course materials;
without an effective sense of purpose, blogs as an assessment can be
perceived as lacking clarity about their function, audience, and tone.
Therefore, the expectations of blog writing should be made clear to
students. Given the necessary alignment with subject-specific degree
outcomes, blog writing may be best understood through a lens of
psychological literacy, given that this approach encourages psychology
students to apply their knowledge to daily life
(\href{https://journals-sagepub-com.gold.idm.oclc.org/doi/full/10.1177/00986283211027278\#bibr17-00986283211027278}{Cranney
et al., 2008},
\href{https://journals-sagepub-com.gold.idm.oclc.org/doi/full/10.1177/00986283211027278\#bibr15-00986283211027278}{2012}).
Therefore, we propose that blog writing may provide a useful opportunity
to allow students to apply their psychology content in a critical,
creative, non-conventional way.

\begin{tcolorbox}[enhanced jigsaw, rightrule=.15mm, opacitybacktitle=0.6, colbacktitle=quarto-callout-note-color!10!white, breakable, leftrule=.75mm, bottomtitle=1mm, toptitle=1mm, colback=white, titlerule=0mm, opacityback=0, title=\textcolor{quarto-callout-note-color}{\faInfo}\hspace{0.5em}{Blog Rubric example}, arc=.35mm, bottomrule=.15mm, toprule=.15mm, left=2mm, coltitle=black]

Scientific Blog marking rubric

Below are the grading criteria for the assessment of the science blog.
The profiles give an

indication of typical performance at each class band, and clearly permit
some variations upwards

or downwards while remaining in the same class band. These descriptions
should be taken as

indicative rather than prescriptive. Seven key attributes of written
work are considered when

assigning marks, with some (e.g., `accuracy') clearly more important
than others:

\begin{enumerate}
\def\labelenumi{\arabic{enumi}.}
\tightlist
\item
  Accuracy (i.e., is the material reported accurately?)
\item
  Appropriateness for general audience (i.e., is key content well
  defined and explained, could a
\item
  non-specialist follow the argument)
\item
  Application to question (i.e., is the material used effectively in the
  assignment)
\item
  Evidence (i.e., are claims supported by relevant evidence and/or
  theory from the literature?)
\item
  Argument (i.e., is there a convincing line of argument through the
  work?)
\item
  Critical evaluation (i.e., is the material presented evaluated fully?)
\item
  Structure and coherence (i.e., is the answer well-structured with good
  flow between points?)
\item
  Presentation (i.e., is the clarity of expression good)
\end{enumerate}

\textbf{See accompanying website for Rubric or go here}
\url{https://osf.io/rgf8t}

\end{tcolorbox}

\bookmarksetup{startatroot}

\hypertarget{considerations}{%
\chapter{Considerations}\label{considerations}}

\bookmarksetup{startatroot}

\hypertarget{further-considerations}{%
\chapter{Further considerations}\label{further-considerations}}

\hypertarget{preparation}{%
\subsection{Preparation}\label{preparation}}

Need to begin preparation

Lectures x 40

Open Educational Resources Textbook for Research Methods CCBY4.0

Lab Practicals x 40

Open Educational Resources Textbook for Data Skills (Navarro) CCBY4.0

Recordings and worksheets for above x 40

Y3/MSc Bootcamp

\hypertarget{infrastructure}{%
\subsection{Infrastructure}\label{infrastructure}}

\hypertarget{recording-suite}{%
\subsubsection{Recording suite}\label{recording-suite}}

\hypertarget{materials-storage}{%
\subsubsection{Materials storage}\label{materials-storage}}

\hypertarget{estates-and-facilities}{%
\subsection{Estates and Facilities}\label{estates-and-facilities}}

\hypertarget{removal-of-computer-banks-in-labs-to-make-them-more-useful-for-practicals}{%
\subsubsection{Removal of computer banks in labs to make them more
useful for
practicals?}\label{removal-of-computer-banks-in-labs-to-make-them-more-useful-for-practicals}}

\hypertarget{wall-mounted-monitors}{%
\subsubsection{Wall-mounted monitors}\label{wall-mounted-monitors}}

\hypertarget{section}{%
\subsubsection{}\label{section}}

\hypertarget{technology}{%
\subsection{Technology}\label{technology}}

\hypertarget{posit-cloud-as-entry-level}{%
\subsubsection{Posit Cloud as Entry
Level}\label{posit-cloud-as-entry-level}}

\hypertarget{student-download-for-y2-onwards}{%
\subsubsection{Student download for Y2
onwards}\label{student-download-for-y2-onwards}}

\hypertarget{possible-posit-server-run-by-ian}{%
\subsubsection{Possible Posit Server run by
Ian}\label{possible-posit-server-run-by-ian}}

\hypertarget{costs}{%
\subsection{Costs}\label{costs}}

\hypertarget{cost-for-posit-cloud-maybe}{%
\subsubsection{Cost for Posit Cloud
(Maybe)}\label{cost-for-posit-cloud-maybe}}

\hypertarget{chromebooks-on-loan}{%
\subsubsection{Chromebooks on loan}\label{chromebooks-on-loan}}

\hypertarget{risks}{%
\subsection{Risks}\label{risks}}

\hypertarget{technology-obstacles-lower-than-spss}{%
\subsubsection{Technology obstacles (lower than
SPSS)}\label{technology-obstacles-lower-than-spss}}

\hypertarget{finite-knowledge-within-staff}{%
\subsubsection{Finite knowledge within
staff}\label{finite-knowledge-within-staff}}

\hypertarget{staff-resistance}{%
\subsubsection{Staff resistance}\label{staff-resistance}}

\hypertarget{staffing}{%
\subsection{Staffing}\label{staffing}}

\hypertarget{recruit-next-tfs-specifically-to-help-build-the-infrastructure-and-programme}{%
\subsubsection{Recruit next TFs specifically to help build the
infrastructure and
programme}\label{recruit-next-tfs-specifically-to-help-build-the-infrastructure-and-programme}}

\hypertarget{timetabling}{%
\subsection{Timetabling}\label{timetabling}}

\hypertarget{heartdata-week}{%
\subsubsection{HeartData week}\label{heartdata-week}}

asdgasdfhg

\hypertarget{induction-planning}{%
\subsubsection{Induction planning}\label{induction-planning}}

\hypertarget{pre-arrival-comms}{%
\subsubsection{Pre-arrival comms}\label{pre-arrival-comms}}

\hypertarget{accessibility}{%
\subsection{Accessibility}\label{accessibility}}

\hypertarget{homework-club---where-staff-are-on-hand-every-week}{%
\subsubsection{Homework club - Where staff are on hand every
week}\label{homework-club---where-staff-are-on-hand-every-week}}

\hypertarget{student-supportwelfare}{%
\subsection{Student Support/Welfare}\label{student-supportwelfare}}

\hypertarget{enhanced-by-access-to-resources}{%
\subsubsection{Enhanced by access to
resources}\label{enhanced-by-access-to-resources}}

\hypertarget{employability}{%
\subsection{Employability}\label{employability}}

\hypertarget{r-and-python-are-most-versatile-tools-on-the-market}{%
\subsubsection{R and Python are most versatile tools on the
market}\label{r-and-python-are-most-versatile-tools-on-the-market}}

\bookmarksetup{startatroot}

\hypertarget{section-1}{%
\chapter{}\label{section-1}}

\part{Years}

\hypertarget{foundations-level-3}{%
\chapter{Foundations (Level 3)}\label{foundations-level-3}}

\hypertarget{section-2}{%
\section{}\label{section-2}}

sdgasdg

\hypertarget{foundation-year-schedule}{%
\chapter{Foundation Year schedule}\label{foundation-year-schedule}}

\begin{longtable}[]{@{}
  >{\centering\arraybackslash}p{(\columnwidth - 8\tabcolsep) * \real{0.1757}}
  >{\raggedright\arraybackslash}p{(\columnwidth - 8\tabcolsep) * \real{0.1757}}
  >{\raggedright\arraybackslash}p{(\columnwidth - 8\tabcolsep) * \real{0.2162}}
  >{\raggedright\arraybackslash}p{(\columnwidth - 8\tabcolsep) * \real{0.1757}}
  >{\raggedright\arraybackslash}p{(\columnwidth - 8\tabcolsep) * \real{0.2568}}@{}}
\toprule()
\begin{minipage}[b]{\linewidth}\centering
Week
\end{minipage} & \begin{minipage}[b]{\linewidth}\raggedright
Schedule
\end{minipage} & \begin{minipage}[b]{\linewidth}\raggedright
\end{minipage} & \begin{minipage}[b]{\linewidth}\raggedright
\end{minipage} & \begin{minipage}[b]{\linewidth}\raggedright
\end{minipage} \\
\midrule()
\endhead
\textbf{1} & \textbf{Lecture:} & There was a wee cooper who lived in
fife and his hat was green & \textbf{IndStud:} & There was a wee cooper
who lived in fife and his hat was green \\
& \textbf{Lab:} & There was a wee cooper who lived in fife and his hat
was green & \textbf{Data:} & There was a wee cooper who lived in fife
and his hat was green \\
\textbf{2} & \textbf{Lecture:} & & \textbf{IndStud:} & \\
& \textbf{Lab:} & & \textbf{Data:} & \\
\textbf{3} & \textbf{Lecture:} & & \textbf{IndStud:} & \\
& \textbf{Lab:} & & \textbf{Data:} & \\
\textbf{4} & \textbf{Lecture:} & & \textbf{IndStud:} & \\
& \textbf{Lab:} & & \textbf{Data:} & \\
\textbf{5} & \textbf{Lecture:} & & \textbf{IndStud:} & \\
& \textbf{Lab:} & & \textbf{Data:} & \\
& & Reading Week & & \\
\textbf{6} & \textbf{Lecture:} & & \textbf{IndStud:} & \\
& \textbf{Lab:} & & \textbf{Data:} & \\
\textbf{7} & \textbf{Lecture:} & & \textbf{IndStud:} & \\
& \textbf{Lab:} & & \textbf{Data:} & \\
\textbf{8} & \textbf{Lecture:} & & \textbf{IndStud:} & \\
& \textbf{Lab:} & & \textbf{Data:} & \\
\textbf{9} & \textbf{Lecture:} & & \textbf{IndStud:} & \\
& \textbf{Lab:} & & \textbf{Data:} & \\
\textbf{10} & \textbf{Lecture:} & & \textbf{IndStud:} & \\
& \textbf{Lab:} & & \textbf{Data:} & \\
\bottomrule()
\end{longtable}

\newpage

\begin{longtable}[]{@{}
  >{\centering\arraybackslash}p{(\columnwidth - 8\tabcolsep) * \real{0.1757}}
  >{\raggedright\arraybackslash}p{(\columnwidth - 8\tabcolsep) * \real{0.1757}}
  >{\raggedright\arraybackslash}p{(\columnwidth - 8\tabcolsep) * \real{0.2162}}
  >{\raggedright\arraybackslash}p{(\columnwidth - 8\tabcolsep) * \real{0.1757}}
  >{\raggedright\arraybackslash}p{(\columnwidth - 8\tabcolsep) * \real{0.2568}}@{}}
\toprule()
\begin{minipage}[b]{\linewidth}\centering
Week
\end{minipage} & \begin{minipage}[b]{\linewidth}\raggedright
Schedule
\end{minipage} & \begin{minipage}[b]{\linewidth}\raggedright
\end{minipage} & \begin{minipage}[b]{\linewidth}\raggedright
\end{minipage} & \begin{minipage}[b]{\linewidth}\raggedright
\end{minipage} \\
\midrule()
\endhead
\textbf{11} & \textbf{Lecture:} & There was a wee cooper who lived in
fife and his hat was green & \textbf{IndStud:} & There was a wee cooper
who lived in fife and his hat was green \\
& \textbf{Lab:} & There was a wee cooper who lived in fife and his hat
was green & \textbf{Data:} & There was a wee cooper who lived in fife
and his hat was green \\
\textbf{12} & \textbf{Lecture:} & & \textbf{IndStud:} & \\
& \textbf{Lab:} & & \textbf{Data:} & \\
\textbf{13} & \textbf{Lecture:} & & \textbf{IndStud:} & \\
& \textbf{Lab:} & & \textbf{Data:} & \\
\textbf{14} & \textbf{Lecture:} & & \textbf{IndStud:} & \\
& \textbf{Lab:} & & \textbf{Data:} & \\
\textbf{15} & \textbf{Lecture:} & & \textbf{IndStud:} & \\
& \textbf{Lab:} & & \textbf{Data:} & \\
& & \textbf{Reading Week} & & \\
\textbf{16} & \textbf{Lecture:} & & \textbf{IndStud:} & \\
& \textbf{Lab:} & & \textbf{Data:} & \\
\textbf{17} & \textbf{Lecture:} & & \textbf{IndStud:} & \\
& \textbf{Lab:} & & \textbf{Data:} & \\
\textbf{18} & \textbf{Lecture:} & & \textbf{IndStud:} & \\
& \textbf{Lab:} & & \textbf{Data:} & \\
\textbf{19} & \textbf{Lecture:} & & \textbf{IndStud:} & \\
& \textbf{Lab:} & & \textbf{Data:} & \\
\textbf{20} & \textbf{Lecture:} & & \textbf{IndStud:} & \\
& \textbf{Lab:} & & \textbf{Data:} & \\
\bottomrule()
\end{longtable}

\hypertarget{year-1-level-4}{%
\chapter{Year 1 (Level 4)}\label{year-1-level-4}}

\hypertarget{year-1-level-4-1}{%
\section{Year 1 (Level 4)}\label{year-1-level-4-1}}

\hypertarget{module-content}{%
\subsection{Module Content}\label{module-content}}

This module equips students with the practical and conceptual skills
necessary for the effective study of psychology. It includes computer
skills, presenting results of experiments, structuring an essay, and
critiquing a scientific paper. Additionally, it provides an introduction
to experimental design, data, and statistics in psychology. Students
will learn the theoretical aspects of basic statistical concepts and
tests, and gain experience using statistical packages.

\hypertarget{module-learning-outcomes}{%
\subsection{Module Learning Outcomes}\label{module-learning-outcomes}}

The student should be able to:

demonstrate a comprehensive understanding of the principles of
experimental psychology, from reading and summarizing scientific papers
to planning, writing and presenting essays, reports and presentations.

understand the importance and relevance of data analysis, the different
types of experiments and tests used.

understand the philosophical underpinnings of qualitative and
quantitative approaches to research and evaluate their merits.

demonstrate the skills to analyse and interpret data using qualitative
and quantitative frameworks and methods.

demonstrate statistical proficiency in the ability to use R to compute
summary statistics, z-scores, chi-square, binomial tests, and parametric
and non-parametric comparison of two means.

be able to visualise and present/communicate research findings to a
range of audiences

select and provide a rationale for using a statistical test to analyse a
particular dataset, and present the results correctly in both graphical
and APA format.

\hypertarget{assessment}{%
\subsection{Assessment}\label{assessment}}

\begin{longtable}[]{@{}lllll@{}}
\toprule()
Assessment Element & Length & \% & F or S & LO Tested \\
\midrule()
\endhead
& & & & \\
& & & & \\
RPS & & & & \\
\bottomrule()
\end{longtable}

\hypertarget{reading-and-resource-list}{%
\subsection{Reading and Resource List}\label{reading-and-resource-list}}

We have a custom made textbook to support key study skills throughout
your degree:

\hypertarget{year-1-schedule}{%
\chapter{Year 1 schedule}\label{year-1-schedule}}

\begin{longtable}[]{@{}
  >{\centering\arraybackslash}p{(\columnwidth - 8\tabcolsep) * \real{0.2000}}
  >{\raggedright\arraybackslash}p{(\columnwidth - 8\tabcolsep) * \real{0.2000}}
  >{\raggedright\arraybackslash}p{(\columnwidth - 8\tabcolsep) * \real{0.2000}}
  >{\raggedright\arraybackslash}p{(\columnwidth - 8\tabcolsep) * \real{0.2000}}
  >{\raggedright\arraybackslash}p{(\columnwidth - 8\tabcolsep) * \real{0.2000}}@{}}
\toprule()
\begin{minipage}[b]{\linewidth}\centering
Week
\end{minipage} & \begin{minipage}[b]{\linewidth}\raggedright
Schedule
\end{minipage} & \begin{minipage}[b]{\linewidth}\raggedright
\end{minipage} & \begin{minipage}[b]{\linewidth}\raggedright
\end{minipage} & \begin{minipage}[b]{\linewidth}\raggedright
\end{minipage} \\
\midrule()
\endhead
\textbf{-1} & \textbf{Pre-Arrival:} & Preparing for Research at
Goldsmiths & \textbf{IndStud:} & Maths ability/anxiety/refresher quiz \\
& \textbf{Lab:} & Online Refresher Q\&A & \textbf{Data:} & \\
\textbf{0} & \textbf{WelcomeWeek} & I❤️Data Fair & \textbf{IndStud:} &
Epistemology/Ontology Task \\
& \textbf{Lab:} & Data Collection on self! & \textbf{Data:} & Reflective
exercise \\
\textbf{1} & \textbf{Lecture:} & Finding patterns and relationships &
\textbf{IndStud:} & First journal article! \\
& \textbf{Lab:} & Visualise the year group & \textbf{Data:} & Blog about
what you have learned. Posit/Quarto \\
\textbf{2} & \textbf{Lecture:} & Finding patterns and relationships &
\textbf{IndStud:} & \\
& \textbf{Lab:} & & \textbf{Data:} & \\
\textbf{3} & \textbf{Lecture:} & Correlations and Models (GLM) &
\textbf{IndStud:} & \\
& \textbf{Lab:} & & \textbf{Data:} & \\
\textbf{4} & \textbf{Lecture:} & Distributions and Sampling &
\textbf{IndStud:} & \\
& \textbf{Lab:} & & \textbf{Data:} & \\
\textbf{5} & \textbf{Lecture:} & Probabilities and P-Values &
\textbf{IndStud:} & \\
& \textbf{Lab:} & & \textbf{Data:} & \\
& & Reading Week & & \\
\textbf{6} & \textbf{Lecture:} & Open Science, Reporting and Critique &
\textbf{IndStud:} & \\
& \textbf{Lab:} & & \textbf{Data:} & \\
\textbf{7} & \textbf{Lecture:} & Qualititative Research &
\textbf{IndStud:} & \\
& \textbf{Lab:} & & \textbf{Data:} & \\
\textbf{8} & \textbf{Lecture:} & Correlational Research &
\textbf{IndStud:} & \\
& \textbf{Lab:} & & \textbf{Data:} & \\
\textbf{9} & \textbf{Lecture:} & Quasi-Experimental Research &
\textbf{IndStud:} & \\
& \textbf{Lab:} & & \textbf{Data:} & \\
\textbf{10} & \textbf{Lecture:} & Experimental Research &
\textbf{IndStud:} & \\
& \textbf{Lab:} & & \textbf{Data:} & \\
\bottomrule()
\end{longtable}

\newpage

\begin{longtable}[]{@{}
  >{\centering\arraybackslash}p{(\columnwidth - 8\tabcolsep) * \real{0.2000}}
  >{\raggedright\arraybackslash}p{(\columnwidth - 8\tabcolsep) * \real{0.2000}}
  >{\raggedright\arraybackslash}p{(\columnwidth - 8\tabcolsep) * \real{0.2000}}
  >{\raggedright\arraybackslash}p{(\columnwidth - 8\tabcolsep) * \real{0.2000}}
  >{\raggedright\arraybackslash}p{(\columnwidth - 8\tabcolsep) * \real{0.2000}}@{}}
\toprule()
\begin{minipage}[b]{\linewidth}\centering
Week
\end{minipage} & \begin{minipage}[b]{\linewidth}\raggedright
Schedule
\end{minipage} & \begin{minipage}[b]{\linewidth}\raggedright
\end{minipage} & \begin{minipage}[b]{\linewidth}\raggedright
\end{minipage} & \begin{minipage}[b]{\linewidth}\raggedright
\end{minipage} \\
\midrule()
\endhead
\textbf{11} & \textbf{Lecture:} & Statistical Models & \textbf{IndStud:}
& There was a wee cooper who lived in fife and his hat was green \\
& \textbf{Lab:} & There was a wee cooper who lived in fife and his hat
was green & \textbf{Data:} & There was a wee cooper who lived in fife
and his hat was green \\
\textbf{12} & \textbf{Lecture:} & Inferential Statistics &
\textbf{IndStud:} & \\
& \textbf{Lab:} & & \textbf{Data:} & \\
\textbf{13} & \textbf{Lecture:} & Alpha, Power, Effect \& Sample Size &
\textbf{IndStud:} & \\
& \textbf{Lab:} & & \textbf{Data:} & \\
\textbf{14} & \textbf{Lecture:} & Correlation in depth &
\textbf{IndStud:} & \\
& \textbf{Lab:} & & \textbf{Data:} & \\
\textbf{15} & \textbf{Lecture:} & Regression & \textbf{IndStud:} & \\
& \textbf{Lab:} & & \textbf{Data:} & \\
& & \textbf{Reading Week} & & \\
\textbf{16} & \textbf{Lecture:} & Multiple Regression &
\textbf{IndStud:} & \\
& \textbf{Lab:} & & \textbf{Data:} & \\
\textbf{17} & \textbf{Lecture:} & Logistic Regression &
\textbf{IndStud:} & \\
& \textbf{Lab:} & & \textbf{Data:} & \\
\textbf{18} & \textbf{Lecture:} & Comparing two means &
\textbf{IndStud:} & \\
& \textbf{Lab:} & & \textbf{Data:} & \\
\textbf{19} & \textbf{Lecture:} & Comparing several means &
\textbf{IndStud:} & \\
& \textbf{Lab:} & & \textbf{Data:} & \\
\textbf{20} & \textbf{Lecture:} & Employability and Data Skills &
\textbf{IndStud:} & \\
& \textbf{Lab:} & & \textbf{Data:} & \\
\bottomrule()
\end{longtable}

\hypertarget{year-2-level-5}{%
\chapter{Year 2 (level 5)}\label{year-2-level-5}}

\hypertarget{year-2-level-5-1}{%
\section{Year 2 (level 5)}\label{year-2-level-5-1}}

\hypertarget{module-aims}{%
\subsection{Module aims}\label{module-aims}}

Experimental designs in psychology typically employ statistical analyses
such as analysis of variance, factor analysis and regression. The aim of
this module is to make these topics more accessible through the use of
practical examples and data collection on a self-directed group research
project.

\hypertarget{module-content-1}{%
\subsection{Module Content}\label{module-content-1}}

The module's overall aim is to offer a supportive and intellectually
rigorous environment allowing students to develop highly valuable,
transferrable research and collaboration skills in the context of
undertaking a group research project.

This module teaches fundamental empirical research techniques within the
framework of Open Science and reproducibility, promoting best practice
in study design, Open Materials and Data, and methodological practice.
This module fully immerses students in the Goldsmiths `community of
practice,' providing structured research support and opportunities to
reflect on learning, modelling the key milestones of the final year
dissertation.

The module seeks to promote the application of a scientific,
intellectually virtuous, research-based approach to any and all future
endeavours, and integrates metacognitive and reflective practices to
deliver this transformative learning towards academic and personal
development.

Over the course of two terms students will follow a programme of
lectures introducing a critical approach to psychological research, as
well as how such skills can be transferred beyond psychology; across
academic disciplines and into the everyday world, with extensive use of
case studies and problem-based learning.

Structured weekly labs will enable students to work collaboratively to
identify an area of research, critically evaluate current research in
the area, and develop a modest research project building on these
insights.

Students will work together, alongside lab tutors and researchers in the
department, to design and deliver the research project, including
obtaining ethical approval, data collection and analysis, then
interpreting and writing up the results, and sharing the materials and
data in line with Open Science best practices in the Psychological,
Behavioural and Data Sciences.

\hypertarget{module-learning-outcomes-1}{%
\subsection{Module Learning Outcomes}\label{module-learning-outcomes-1}}

\begin{enumerate}
\def\labelenumi{\arabic{enumi}.}
\item
  Show a critical understanding of research design and methodology
\item
  Design, conduct, analyse, interpret and disseminate a psychological
  research project
\item
  Understand the conceptual and historical issues concerned with
  psychology as a science and area of practical application
\item
  Demonstrate valuable time-management and collaborative
  project-management skills and proficiencies
\item
  Reflect on their own learning, skill development and metacognition,
  preparing them for the final year dissertation
\item
  be able to use R to analyse: regression, correlations, reliability and
  validity, effect sizes, one-way within and between subjects designs
  (and post-hocs), two-way within, between and mixed designs; and factor
  analysis
\item
  Be able to present reproducible, APA format literate-programmed
  research reports.
\end{enumerate}

\hypertarget{assessment-1}{%
\subsection{Assessment}\label{assessment-1}}

\begin{longtable}[]{@{}lllll@{}}
\toprule()
Assessment Element & Length & \% & F or S & LO Tested \\
\midrule()
\endhead
& & & & \\
& & & & \\
RPS & & & & \\
\bottomrule()
\end{longtable}

\hypertarget{reading-and-resource-list-1}{%
\subsection{Reading and Resource
List}\label{reading-and-resource-list-1}}

We have a custom made textbook to support key study skills throughout
your degree:

\hypertarget{year-2-schedule}{%
\chapter{Year 2 schedule}\label{year-2-schedule}}

\begin{longtable}[]{@{}
  >{\centering\arraybackslash}p{(\columnwidth - 8\tabcolsep) * \real{0.2000}}
  >{\raggedright\arraybackslash}p{(\columnwidth - 8\tabcolsep) * \real{0.2000}}
  >{\raggedright\arraybackslash}p{(\columnwidth - 8\tabcolsep) * \real{0.2000}}
  >{\raggedright\arraybackslash}p{(\columnwidth - 8\tabcolsep) * \real{0.2000}}
  >{\raggedright\arraybackslash}p{(\columnwidth - 8\tabcolsep) * \real{0.2000}}@{}}
\toprule()
\begin{minipage}[b]{\linewidth}\centering
Week
\end{minipage} & \begin{minipage}[b]{\linewidth}\raggedright
Schedule
\end{minipage} & \begin{minipage}[b]{\linewidth}\raggedright
\end{minipage} & \begin{minipage}[b]{\linewidth}\raggedright
\end{minipage} & \begin{minipage}[b]{\linewidth}\raggedright
\end{minipage} \\
\midrule()
\endhead
\textbf{1} & \textbf{Lecture:} & ANOVA recap & \textbf{IndStud:} & There
was a wee cooper who lived in fife and his hat was green \\
& \textbf{Lab:} & There was a wee cooper who lived in fife and his hat
was green & \textbf{Data:} & There was a wee cooper who lived in fife
and his hat was green \\
\textbf{2} & \textbf{Lecture:} & ANCOVA & \textbf{IndStud:} & \\
& \textbf{Lab:} & & \textbf{Data:} & \\
\textbf{3} & \textbf{Lecture:} & Factorial ANOVA & \textbf{IndStud:}
& \\
& \textbf{Lab:} & & \textbf{Data:} & \\
\textbf{4} & \textbf{Lecture:} & RM ANOVA & \textbf{IndStud:} & \\
& \textbf{Lab:} & & \textbf{Data:} & \\
\textbf{5} & \textbf{Lecture:} & Mixed Designs & \textbf{IndStud:} & \\
& \textbf{Lab:} & & \textbf{Data:} & \\
& & Reading Week & & \\
\textbf{6} & \textbf{Lecture:} & Non-parametrics and non-numeric data &
\textbf{IndStud:} & \\
& \textbf{Lab:} & & \textbf{Data:} & \\
\textbf{7} & \textbf{Lecture:} & Case studies, n=1 \& Ethnography &
\textbf{IndStud:} & \\
& \textbf{Lab:} & & \textbf{Data:} & \\
\textbf{8} & \textbf{Lecture:} & Thematic Analysis \& Grounded Theory &
\textbf{IndStud:} & \\
& \textbf{Lab:} & & \textbf{Data:} & \\
\textbf{9} & \textbf{Lecture:} & Focus Groups \& Consumer Research &
\textbf{IndStud:} & \\
& \textbf{Lab:} & & \textbf{Data:} & \\
\textbf{10} & \textbf{Lecture:} & Interpretative Phenomenological
Analysis \& Discourse Analysis & \textbf{IndStud:} & \\
& \textbf{Lab:} & & \textbf{Data:} & \\
\bottomrule()
\end{longtable}

\newpage

\begin{longtable}[]{@{}
  >{\centering\arraybackslash}p{(\columnwidth - 8\tabcolsep) * \real{0.2000}}
  >{\raggedright\arraybackslash}p{(\columnwidth - 8\tabcolsep) * \real{0.2000}}
  >{\raggedright\arraybackslash}p{(\columnwidth - 8\tabcolsep) * \real{0.2000}}
  >{\raggedright\arraybackslash}p{(\columnwidth - 8\tabcolsep) * \real{0.2000}}
  >{\raggedright\arraybackslash}p{(\columnwidth - 8\tabcolsep) * \real{0.2000}}@{}}
\toprule()
\begin{minipage}[b]{\linewidth}\centering
Week
\end{minipage} & \begin{minipage}[b]{\linewidth}\raggedright
Schedule
\end{minipage} & \begin{minipage}[b]{\linewidth}\raggedright
\end{minipage} & \begin{minipage}[b]{\linewidth}\raggedright
\end{minipage} & \begin{minipage}[b]{\linewidth}\raggedright
\end{minipage} \\
\midrule()
\endhead
\textbf{11} & \textbf{Lecture:} & Your final year project &
\textbf{IndStud:} & There was a wee cooper who lived in fife and his hat
was green \\
& \textbf{Lab:} & There was a wee cooper who lived in fife and his hat
was green & \textbf{Data:} & There was a wee cooper who lived in fife
and his hat was green \\
\textbf{12} & \textbf{Lecture:} & MANOVA & \textbf{IndStud:} & \\
& \textbf{Lab:} & & \textbf{Data:} & \\
\textbf{13} & \textbf{Lecture:} & FA (LVM, PCA, EFA, CFA) &
\textbf{IndStud:} & \\
& \textbf{Lab:} & & \textbf{Data:} & \\
\textbf{14} & \textbf{Lecture:} & Mediation \& Moderation &
\textbf{IndStud:} & \\
& \textbf{Lab:} & & \textbf{Data:} & \\
\textbf{15} & \textbf{Lecture:} & Longitudinal Data & \textbf{IndStud:}
& \\
& \textbf{Lab:} & & \textbf{Data:} & \\
& & \textbf{Reading Week} & & \\
\textbf{16} & \textbf{Lecture:} & Multi-Level Models & \textbf{IndStud:}
& \\
& \textbf{Lab:} & & \textbf{Data:} & \\
\textbf{17} & \textbf{Lecture:} & Categorical Data & \textbf{IndStud:}
& \\
& \textbf{Lab:} & & \textbf{Data:} & \\
\textbf{18} & \textbf{Lecture:} & Big Data & \textbf{IndStud:} & \\
& \textbf{Lab:} & & \textbf{Data:} & \\
\textbf{19} & \textbf{Lecture:} & Machine Learning & \textbf{IndStud:}
& \\
& \textbf{Lab:} & & \textbf{Data:} & \\
\textbf{20} & \textbf{Lecture:} & Artificial Intelligence &
\textbf{IndStud:} & \\
& \textbf{Lab:} & & \textbf{Data:} & \\
\bottomrule()
\end{longtable}

\hypertarget{dissertation-y3-msc}{%
\chapter{Dissertation (Y3 \& MSc)}\label{dissertation-y3-msc}}

\hypertarget{level-6---topline-summary}{%
\section{Level 6 - topline summary}\label{level-6---topline-summary}}

\hypertarget{module-content-2}{%
\subsection{Module Content}\label{module-content-2}}

\hypertarget{module-learning-outcomes-2}{%
\subsection{Module Learning Outcomes}\label{module-learning-outcomes-2}}

\hypertarget{assessment-2}{%
\subsection{Assessment}\label{assessment-2}}

\begin{longtable}[]{@{}lllll@{}}
\toprule()
Assessment Element & Length & \% & F or S & LO Tested \\
\midrule()
\endhead
& & & & \\
& & & & \\
& & & & \\
\bottomrule()
\end{longtable}

\hypertarget{reading-and-resource-list-2}{%
\subsection{Reading and Resource
List}\label{reading-and-resource-list-2}}

We have a custom made textbook to support key study skills throughout
your degree:

\newpage

\begin{longtable}[]{@{}ccc@{}}
\caption{Y3 Term 1 Laydown}\tabularnewline
\toprule()
Week & Lecture & Practical \\
\midrule()
\endfirsthead
\toprule()
Week & Lecture & Practical \\
\midrule()
\endhead
Pre & Preparing to become a Psychologist & \\
WW & Let's measure some stuff & \\
1 & Answering questions with data & Doing stuff with stuff \\
2 & Finding patterns and relationships & \\
3 & Correlations and models & \\
4 & Distributions and sampling & \\
5 & Probabilities and P-Values & \\
RW & -\/- & -\/- \\
\bottomrule()
\end{longtable}

\hypertarget{dissertation-schedule-y3-msc}{%
\chapter{Dissertation Schedule (Y3 \&
MSc)}\label{dissertation-schedule-y3-msc}}

\begin{longtable}[]{@{}
  >{\centering\arraybackslash}p{(\columnwidth - 8\tabcolsep) * \real{0.1757}}
  >{\raggedright\arraybackslash}p{(\columnwidth - 8\tabcolsep) * \real{0.1757}}
  >{\raggedright\arraybackslash}p{(\columnwidth - 8\tabcolsep) * \real{0.2162}}
  >{\raggedright\arraybackslash}p{(\columnwidth - 8\tabcolsep) * \real{0.1757}}
  >{\raggedright\arraybackslash}p{(\columnwidth - 8\tabcolsep) * \real{0.2568}}@{}}
\toprule()
\begin{minipage}[b]{\linewidth}\centering
Week
\end{minipage} & \begin{minipage}[b]{\linewidth}\raggedright
Schedule
\end{minipage} & \begin{minipage}[b]{\linewidth}\raggedright
\end{minipage} & \begin{minipage}[b]{\linewidth}\raggedright
\end{minipage} & \begin{minipage}[b]{\linewidth}\raggedright
\end{minipage} \\
\midrule()
\endhead
\textbf{1} & \textbf{Lecture:} & There was a wee cooper who lived in
fife and his hat was green & \textbf{IndStud:} & There was a wee cooper
who lived in fife and his hat was green \\
& \textbf{Lab:} & There was a wee cooper who lived in fife and his hat
was green & \textbf{Data:} & There was a wee cooper who lived in fife
and his hat was green \\
\textbf{2} & \textbf{Lecture:} & & \textbf{IndStud:} & \\
& \textbf{Lab:} & & \textbf{Data:} & \\
\textbf{3} & \textbf{Lecture:} & & \textbf{IndStud:} & \\
& \textbf{Lab:} & & \textbf{Data:} & \\
\textbf{4} & \textbf{Lecture:} & & \textbf{IndStud:} & \\
& \textbf{Lab:} & & \textbf{Data:} & \\
\textbf{5} & \textbf{Lecture:} & & \textbf{IndStud:} & \\
& \textbf{Lab:} & & \textbf{Data:} & \\
& & Reading Week & & \\
\bottomrule()
\end{longtable}

\hypertarget{year-3-rm-module-new-module-level-6}{%
\chapter{Year 3 RM module (NEW MODULE) (Level
6)}\label{year-3-rm-module-new-module-level-6}}

\hypertarget{year-3-level-6}{%
\section{Year 3 (Level 6)}\label{year-3-level-6}}

\hypertarget{module-content-3}{%
\subsection{Module Content}\label{module-content-3}}

\hypertarget{module-learning-outcomes-3}{%
\subsection{Module Learning Outcomes}\label{module-learning-outcomes-3}}

\hypertarget{assessment-3}{%
\subsection{Assessment}\label{assessment-3}}

\begin{longtable}[]{@{}lllll@{}}
\toprule()
Assessment Element & Length & \% & F or S & LO Tested \\
\midrule()
\endhead
& & & & \\
& & & & \\
& & & & \\
\bottomrule()
\end{longtable}

\hypertarget{reading-and-resource-list-3}{%
\subsection{Reading and Resource
List}\label{reading-and-resource-list-3}}

We have a custom made textbook to support key study skills throughout
your degree:

\hypertarget{year-3-schedule-new-module}{%
\chapter{Year 3 schedule (NEW
MODULE)}\label{year-3-schedule-new-module}}

\begin{longtable}[]{@{}
  >{\centering\arraybackslash}p{(\columnwidth - 8\tabcolsep) * \real{0.1757}}
  >{\raggedright\arraybackslash}p{(\columnwidth - 8\tabcolsep) * \real{0.1757}}
  >{\raggedright\arraybackslash}p{(\columnwidth - 8\tabcolsep) * \real{0.2162}}
  >{\raggedright\arraybackslash}p{(\columnwidth - 8\tabcolsep) * \real{0.1757}}
  >{\raggedright\arraybackslash}p{(\columnwidth - 8\tabcolsep) * \real{0.2568}}@{}}
\toprule()
\begin{minipage}[b]{\linewidth}\centering
Week
\end{minipage} & \begin{minipage}[b]{\linewidth}\raggedright
Schedule
\end{minipage} & \begin{minipage}[b]{\linewidth}\raggedright
\end{minipage} & \begin{minipage}[b]{\linewidth}\raggedright
\end{minipage} & \begin{minipage}[b]{\linewidth}\raggedright
\end{minipage} \\
\midrule()
\endhead
\textbf{1} & \textbf{Lecture:} & There was a wee cooper who lived in
fife and his hat was green & \textbf{IndStud:} & There was a wee cooper
who lived in fife and his hat was green \\
& \textbf{Lab:} & There was a wee cooper who lived in fife and his hat
was green & \textbf{Data:} & There was a wee cooper who lived in fife
and his hat was green \\
\textbf{2} & \textbf{Lecture:} & & \textbf{IndStud:} & \\
& \textbf{Lab:} & & \textbf{Data:} & \\
\textbf{3} & \textbf{Lecture:} & & \textbf{IndStud:} & \\
& \textbf{Lab:} & & \textbf{Data:} & \\
\textbf{4} & \textbf{Lecture:} & & \textbf{IndStud:} & \\
& \textbf{Lab:} & & \textbf{Data:} & \\
\textbf{5} & \textbf{Lecture:} & & \textbf{IndStud:} & \\
& \textbf{Lab:} & & \textbf{Data:} & \\
& & Reading Week & & \\
\textbf{6} & \textbf{Lecture:} & & \textbf{IndStud:} & \\
& \textbf{Lab:} & & \textbf{Data:} & \\
\textbf{7} & \textbf{Lecture:} & & \textbf{IndStud:} & \\
& \textbf{Lab:} & & \textbf{Data:} & \\
\textbf{8} & \textbf{Lecture:} & & \textbf{IndStud:} & \\
& \textbf{Lab:} & & \textbf{Data:} & \\
\textbf{9} & \textbf{Lecture:} & & \textbf{IndStud:} & \\
& \textbf{Lab:} & & \textbf{Data:} & \\
\textbf{10} & \textbf{Lecture:} & & \textbf{IndStud:} & \\
& \textbf{Lab:} & & \textbf{Data:} & \\
\bottomrule()
\end{longtable}

\hypertarget{msc-module-new}{%
\chapter{MSc Module (NEW)}\label{msc-module-new}}

xcghfg

\hypertarget{msc-module-new-schedule}{%
\chapter{MSc Module (NEW) schedule}\label{msc-module-new-schedule}}

\begin{longtable}[]{@{}
  >{\centering\arraybackslash}p{(\columnwidth - 8\tabcolsep) * \real{0.1757}}
  >{\raggedright\arraybackslash}p{(\columnwidth - 8\tabcolsep) * \real{0.1757}}
  >{\raggedright\arraybackslash}p{(\columnwidth - 8\tabcolsep) * \real{0.2162}}
  >{\raggedright\arraybackslash}p{(\columnwidth - 8\tabcolsep) * \real{0.1757}}
  >{\raggedright\arraybackslash}p{(\columnwidth - 8\tabcolsep) * \real{0.2568}}@{}}
\toprule()
\begin{minipage}[b]{\linewidth}\centering
Week
\end{minipage} & \begin{minipage}[b]{\linewidth}\raggedright
Schedule
\end{minipage} & \begin{minipage}[b]{\linewidth}\raggedright
\end{minipage} & \begin{minipage}[b]{\linewidth}\raggedright
\end{minipage} & \begin{minipage}[b]{\linewidth}\raggedright
\end{minipage} \\
\midrule()
\endhead
\textbf{1} & \textbf{Lecture:} & There was a wee cooper who lived in
fife and his hat was green & \textbf{IndStud:} & There was a wee cooper
who lived in fife and his hat was green \\
& \textbf{Lab:} & There was a wee cooper who lived in fife and his hat
was green & \textbf{Data:} & There was a wee cooper who lived in fife
and his hat was green \\
\textbf{2} & \textbf{Lecture:} & & \textbf{IndStud:} & \\
& \textbf{Lab:} & & \textbf{Data:} & \\
\textbf{3} & \textbf{Lecture:} & & \textbf{IndStud:} & \\
& \textbf{Lab:} & & \textbf{Data:} & \\
\textbf{4} & \textbf{Lecture:} & & \textbf{IndStud:} & \\
& \textbf{Lab:} & & \textbf{Data:} & \\
\textbf{5} & \textbf{Lecture:} & & \textbf{IndStud:} & \\
& \textbf{Lab:} & & \textbf{Data:} & \\
& & Reading Week & & \\
\textbf{6} & \textbf{Lecture:} & & \textbf{IndStud:} & \\
& \textbf{Lab:} & & \textbf{Data:} & \\
\textbf{7} & \textbf{Lecture:} & & \textbf{IndStud:} & \\
& \textbf{Lab:} & & \textbf{Data:} & \\
\textbf{8} & \textbf{Lecture:} & & \textbf{IndStud:} & \\
& \textbf{Lab:} & & \textbf{Data:} & \\
\textbf{9} & \textbf{Lecture:} & & \textbf{IndStud:} & \\
& \textbf{Lab:} & & \textbf{Data:} & \\
\textbf{10} & \textbf{Lecture:} & & \textbf{IndStud:} & \\
& \textbf{Lab:} & & \textbf{Data:} & \\
\bottomrule()
\end{longtable}

\newpage

\begin{longtable}[]{@{}
  >{\centering\arraybackslash}p{(\columnwidth - 8\tabcolsep) * \real{0.1757}}
  >{\raggedright\arraybackslash}p{(\columnwidth - 8\tabcolsep) * \real{0.1757}}
  >{\raggedright\arraybackslash}p{(\columnwidth - 8\tabcolsep) * \real{0.2162}}
  >{\raggedright\arraybackslash}p{(\columnwidth - 8\tabcolsep) * \real{0.1757}}
  >{\raggedright\arraybackslash}p{(\columnwidth - 8\tabcolsep) * \real{0.2568}}@{}}
\toprule()
\begin{minipage}[b]{\linewidth}\centering
Week
\end{minipage} & \begin{minipage}[b]{\linewidth}\raggedright
Schedule
\end{minipage} & \begin{minipage}[b]{\linewidth}\raggedright
\end{minipage} & \begin{minipage}[b]{\linewidth}\raggedright
\end{minipage} & \begin{minipage}[b]{\linewidth}\raggedright
\end{minipage} \\
\midrule()
\endhead
\textbf{11} & \textbf{Lecture:} & There was a wee cooper who lived in
fife and his hat was green & \textbf{IndStud:} & There was a wee cooper
who lived in fife and his hat was green \\
& \textbf{Lab:} & There was a wee cooper who lived in fife and his hat
was green & \textbf{Data:} & There was a wee cooper who lived in fife
and his hat was green \\
\textbf{12} & \textbf{Lecture:} & & \textbf{IndStud:} & \\
& \textbf{Lab:} & & \textbf{Data:} & \\
\textbf{13} & \textbf{Lecture:} & & \textbf{IndStud:} & \\
& \textbf{Lab:} & & \textbf{Data:} & \\
\textbf{14} & \textbf{Lecture:} & & \textbf{IndStud:} & \\
& \textbf{Lab:} & & \textbf{Data:} & \\
\textbf{15} & \textbf{Lecture:} & & \textbf{IndStud:} & \\
& \textbf{Lab:} & & \textbf{Data:} & \\
& & \textbf{Reading Week} & & \\
\textbf{16} & \textbf{Lecture:} & & \textbf{IndStud:} & \\
& \textbf{Lab:} & & \textbf{Data:} & \\
\textbf{17} & \textbf{Lecture:} & & \textbf{IndStud:} & \\
& \textbf{Lab:} & & \textbf{Data:} & \\
\textbf{18} & \textbf{Lecture:} & & \textbf{IndStud:} & \\
& \textbf{Lab:} & & \textbf{Data:} & \\
\textbf{19} & \textbf{Lecture:} & & \textbf{IndStud:} & \\
& \textbf{Lab:} & & \textbf{Data:} & \\
\textbf{20} & \textbf{Lecture:} & & \textbf{IndStud:} & \\
& \textbf{Lab:} & & \textbf{Data:} & \\
\bottomrule()
\end{longtable}

\part{Guidelines}

\hypertarget{qaa}{%
\chapter{QAA}\label{qaa}}

\hypertarget{qaa-benchmarks}{%
\chapter{QAA Benchmarks}\label{qaa-benchmarks}}

3.4 Research methods are integral to psychology and students obtain a
sound knowledge of, and a proven ability to use, a range of both
qualitative and quantitative methods appropriately. Knowledge and
understanding of how to obtain and analyse evidence is best acquired and
demonstrated through extensive and progressive empirical work in
laboratory and naturalistic settings through all stages of a degree.

3.5 Psychology students learn the basic principles of sound data
collection. Given the broad theoretical scope of psychology, rigorous
specialist training is required to engender a critical understanding of
the role of experimental design, the choice of research methods
employed, and the analytic approach taken, for testing psychological
theories.

Subject knowledge and understanding 6.3 On graduating with an honours
degree in psychology, graduates are able to:

\begin{enumerate}
\def\labelenumi{\arabic{enumi}.}
\item
  understand the scientific underpinnings of psychology as a discipline,
  its historical origins, development and limitations
\item
  recognise the inherent variability and diversity of psychological
  functioning and its significance
\item
  demonstrate systematic knowledge and critical understanding of a range
  of influences on psychological functioning, how they are
  conceptualised across the core areas as outlined in paragraphs 4.4 and
  4.5 and how they interrelate
\item
  demonstrate detailed knowledge of several specialised areas and/or
  applications, some of which are at the cutting edge of research in the
  discipline
\item
  demonstrate a systematic knowledge of a range of research paradigms,
  research methods and measurement techniques, including statistics and
  probability, and be aware of their limitations.
\end{enumerate}

Subject-specific skills 6.4 On graduating with an honours degree in
psychology, graduates are able to:

\begin{enumerate}
\def\labelenumi{\arabic{enumi}.}
\item
  reason scientifically, understand the role of evidence and make
  critical judgements about arguments in psychology
\item
  adopt multiple perspectives and systematically analyse the
  relationships between them
\item
  detect meaningful patterns in behaviour and evaluate their
  significance
\item
  recognise the subjective and variable nature of individual experience
\item
  pose, operationalise and critique research questions
\item
  demonstrate substantial competence in research skills through
  practical activities
\item
  reason analytically and demonstrate competence in a range of
  quantitative and qualitative methods
\item
  competently initiate, design, conduct and report on an
  empirically-based research project under appropriate supervision, and
  recognise its theoretical, practical and methodological implications
  and limitations
\item
  be aware of ethical principles and approval procedures and demonstrate
  these in relation to personal study, particularly with regard to the
  research project, and be aware of the ethical context of psychology as
  a discipline.
\end{enumerate}

Generic skills 6.5 On graduating with an honours degree in psychology,
graduates are able to:

\begin{enumerate}
\def\labelenumi{\arabic{enumi}.}
\item
  \textbf{communicate ideas and research findings by written, oral and
  visual means}
\item
  \textbf{interpret and use numerical, textual and other forms of data}
\item
  \textbf{be computer literate, for the purposes of furthering their own
  learning and in the analysis and presentation of ideas and research
  findings}
\item
  \textbf{solve problems by clarifying questions, considering
  alternative solutions and evaluating outcomes}
\item
  \textbf{be sensitive to, and take account of, contextual and
  interpersonal factors in groups and teams}
\item
  \textbf{undertake self-directed study and project management, in order
  to meet desired objectives}
\item
  \textbf{take charge of their own learning, and reflect and evaluate
  personal strengths and weaknesses for the purposes of future
  learning.}
\end{enumerate}

\hypertarget{references}{%
\chapter*{References}\label{references}}
\addcontentsline{toc}{chapter}{References}

\markboth{References}{References}

\hypertarget{refs}{}
\begin{CSLReferences}{1}{0}
\leavevmode\vadjust pre{\hypertarget{ref-archbald1991}{}}%
Archbald, D. A. (1991). Authentic assessment: Principles, practices, and
issues. \emph{School Psychology Quarterly}, \emph{6}(4), 279--293.
\url{https://doi.org/10.1037/h0088821}

\leavevmode\vadjust pre{\hypertarget{ref-macandrew2002}{}}%
Macandrew, S. B. G., \& Edwards, K. (2002). Essays are Not the Only Way:
A Case Report on the Benefits of Authentic Assessment. \emph{Psychology
Learning \& Teaching}, \emph{2}(2), 134--139.
\url{https://doi.org/10.2304/plat.2002.2.2.134}

\leavevmode\vadjust pre{\hypertarget{ref-pownall2023}{}}%
Pownall, M., Havelka, J., \& Harris, R. (2023). Scientific blogs as a
psychological literacy assessment tool. \emph{Teaching of Psychology},
\emph{50}(1), 69--76. \url{https://doi.org/10.1177/00986283211027278}

\leavevmode\vadjust pre{\hypertarget{ref-sathy2020}{}}%
Sathy, V., \& Moore, Q. (2020). Who benefits from the flipped
classroom?: Quasi-experimental findings on student learning, engagement,
course perceptions, and interest in statistics. In \emph{Teaching
statistics and quantitative methods in the 21st century}. Routledge.

\leavevmode\vadjust pre{\hypertarget{ref-sokhanvar2021}{}}%
Sokhanvar, Z., Salehi, K., \& Sokhanvar, F. (2021). Advantages of
authentic assessment for improving the learning experience and
employability skills of higher education students: A systematic
literature review. \emph{Studies in Educational Evaluation}, \emph{70},
101030. \url{https://doi.org/10.1016/j.stueduc.2021.101030}

\leavevmode\vadjust pre{\hypertarget{ref-archbald1991}{}}%
Archbald, D. A. (1991). Authentic assessment: Principles, practices, and
issues. \emph{School Psychology Quarterly}, \emph{6}(4), 279--293.
\url{https://doi.org/10.1037/h0088821}

\leavevmode\vadjust pre{\hypertarget{ref-macandrew2002}{}}%
Macandrew, S. B. G., \& Edwards, K. (2002). Essays are Not the Only Way:
A Case Report on the Benefits of Authentic Assessment. \emph{Psychology
Learning \& Teaching}, \emph{2}(2), 134--139.
\url{https://doi.org/10.2304/plat.2002.2.2.134}

\leavevmode\vadjust pre{\hypertarget{ref-pownall2023}{}}%
Pownall, M., Havelka, J., \& Harris, R. (2023). Scientific blogs as a
psychological literacy assessment tool. \emph{Teaching of Psychology},
\emph{50}(1), 69--76. \url{https://doi.org/10.1177/00986283211027278}

\leavevmode\vadjust pre{\hypertarget{ref-sathy2020}{}}%
Sathy, V., \& Moore, Q. (2020). Who benefits from the flipped
classroom?: Quasi-experimental findings on student learning, engagement,
course perceptions, and interest in statistics. In \emph{Teaching
statistics and quantitative methods in the 21st century}. Routledge.

\leavevmode\vadjust pre{\hypertarget{ref-sokhanvar2021}{}}%
Sokhanvar, Z., Salehi, K., \& Sokhanvar, F. (2021). Advantages of
authentic assessment for improving the learning experience and
employability skills of higher education students: A systematic
literature review. \emph{Studies in Educational Evaluation}, \emph{70},
101030. \url{https://doi.org/10.1016/j.stueduc.2021.101030}

\end{CSLReferences}


\backmatter

\end{document}
